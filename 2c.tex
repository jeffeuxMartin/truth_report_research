\section{本章節總結}

本章節先是對作為 building block 的類神經網路進行了基本原理的介紹,其後對本論文研究的核心 --- 「representation」與「discrete unit」的發展演進與歷史進行了簡單的梳理。此後兩章節就會緊扣著這些基石模型得到的離散特徵,將其與尤其是 phoneme 這類語音學標記之間的統計關係進行更進一步的分析。

(Ch 3 好像要講一下為什麼不做連續特徵 \& 為什麼要以 phoneme 為客體了 T\_T)


 