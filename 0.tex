
% 第二塊 %%%%%%%%%%%%%%%%%%

% 第一塊 %%%%%%%%%%%%%%%%%%
%%%%%% \subsection{基於各自音位的分析}

%%%%%%%%% (放數據)

\begin{align}
    
\end{align}

% 第一塊後半 %%%%%%%%%%%%%%%%%%









% 後面寫 arpabet?

% 調音部位 →    唇   舌冠  舌背  咽喉
% 調音方法 ↓    雙唇  唇齒  舌唇  齒   齒齦  齦後  捲舌  齦腭  硬腭  軟腭  小舌  咽   會⁠厭 聲門
% 鼻音    m̥  m       ɱ       n̼          n̥  n           ɳ̊  ɳ           ɲ̊  ɲ   ŋ̊  ŋ       ɴ                       
% 塞音    p   b   p̪  b̪  t̼  d̼          t   d           ʈ   ɖ           c   ɟ   k   ɡ   q   ɢ           ʡ       ʔ   
% 有噝擦音                                  s   z   ʃ   ʒ   ʂ   ʐ   ɕ   ʑ                                               
% 無噝擦音  ɸ   β   f   v   θ̼  ð̼  θ   ð   θ̠  ð̠  ɹ̠̊˔    ɹ̠˔     ɻ˔          ç   ʝ   x   ɣ   χ   ʁ   ħ   ʕ   ʜ   ʢ   h   ɦ
% 近音            ʋ̥  ʋ                   ɹ̥  ɹ           ɻ̊  ɻ           j̊  j   ɰ̊  ɰ                               ʔ̞
% 閃音        ⱱ̟      ⱱ       ɾ̼          ɾ̥  ɾ           ɽ̊  ɽ                               ɢ̆              ʡ̆      
% 顫音    ʙ̥  ʙ                           r̥  r           ɽ̊r̥    ɽr                          ʀ̥  ʀ                       
% 邊擦音                                   ɬ   ɮ           ɭ̊˔ ɭ˔          ʎ̝̊ ʎ̝  ʟ̝̊ ʟ̝                              
% 邊近音                                   l̥  l           ɭ̊  ɭ           ʎ̥  ʎ   ʟ̥  ʟ       ʟ̠                      
% 邊閃音                                       ɺ               ɭ̆              ʎ̆      ʟ̆                              
% 國際音標說明國際音標表模板







%%%% BETTER Have Todo!有機率看哪裡說一下 HuBERT……
%%%% BETTER Have Todo!有機率看哪裡說一下 HuBERT……
%%%% BETTER Have Todo!有機率看哪裡說一下 HuBERT……



%%%%%% \section{語音學分類(phone type)}
%%%%%% 
%%%%%% \subsection{簡介}
%%%%%% 
%%%%%% 除了單一 phn 本身的特性以外,由於 phn 本身彼此不是完全獨立的,而是彼此之間就存在相似的特徵,可以分成幾個組別。因此,依照 CITEME (tanghao等三篇) 的分組方式,對英語的 phn 進行分類並合併比對數據,看看這些 unit 本身是否有 capture 到相似的發聲特徵,而不單純只是把 phn 分成約五十類完全獨立的標籤。
%%%%%% 
%%%%%% % (基於語音表徵本身就是 acoustic sisgnals 來的,應該 by nature 要可以對語音特徵分組吧?)
%%%%%% 
%%%%%% % 以下為各分組進行簡單介紹:
%%%%%% 
%%%%%% % \subsubsection{consonants}
%%%%%% 
%%%%%% 按照發音的方式,子音可分為五類:
%%%%%% 
%%%%%% \begin{itemize}
%%%%%%     \item (Plosive)
%%%%%%     \item 擦音(Fricative)
%%%%%%     \item 塞擦音(Affircate)
%%%%%%     \item 鼻音(Nasal)
%%%%%%     \item 近音(Approximant)
%%%%%% \end{itemize}
%%%%%% 
%%%%%% % \subsubsection{vowels}
%%%%%% 
%%%%%% 母音則是被分為:
%%%%%% 
%%%%%% \begin{itemize}
%%%%%%     \item 單原因
%%%%%%     \item 雙原因
%%%%%% \end{itemize}
%%%%%% 
%%%%%% % 母音在這邊為了簡單起見,會被分在一起?
%%%%%% 
%%%%%% \subsection{解釋意義}
%%%%%% 
%%%%%% \begin{itemize}
%%%%%%     \item 純度(purity):換成type之後有何變化(關聯性更強?)
%%%%%%     \item 熵(entropy)(放直方圖解釋) --> phone type更明顯?
%%%%%%     \item 對齊(alignment):是否減少segment資訊的保留(連續子音母音被合併?
%%%%%% \end{itemize}



%%%%%% 
%%%%%% \subsection{基於語音學分類的分析}
%%%%%% 
%%%%%% (放數據)

%%%%%%%%%%%%%%%%%% \subsection{以語音分段指標衡量對齊(Alignment)程度}
%%%%%%%%%%%%%%%%%%   
%%%%%%%%%%%%%%%%%% 為了分析離散單元跟音位在語句序列之間的對齊程度,
%%%%%%%%%%%%%%%%%% 本研究根據 \cite{strgar_phoneme_2023} 的方法
%%%%%%%%%%%%%%%%%% 採用語音分段(Speech Segmentation)的標準去衡量。
%%%%%%%%%%%%%%%%%% 具體動機為將被分到同一個離散單元編號的音框當成語音分段的同一類別的音位,
%%%%%%%%%%%%%%%%%% 以此
%%%%%%%%%%%%%%%%%% 期望可以
%%%%%%%%%%%%%%%%%% 觀察出
%%%%%%%%%%%%%%%%%% 在每一段語句中,
%%%%%%%%%%%%%%%%%% 離散單元出現的順序與範圍,
%%%%%%%%%%%%%%%%%% 與音位標註指示的範圍一致的程度。
%%%%%%%%%%%%%%%%%% 
%%%%%%%%%%%%%%%%%% %% alignment

% 看 alignment 並 cite 那邊,說明一下
% 寫子母音

%%%%%%%%% 糟糕,沒講 HuBERT 但隨便
