

        ()
七歲有存在語音上香蕉
        論文透過對離散單元與對齊進行分詞方法並且驗證在程度與相互資訊等分析數據的比對之下對於目前無文字架構中提及的四種語音表徵進行比對與分析。

        。本論文。透過七歲有存在語音上香蕉

%%%%%%%%%%%%%%%%%%%%%%%%%%%%%%
%%%%%%%%%%%%%%%%%%%%%%%%%%%%%%

        在語音處理的領域中,特徵抽取(feature extraction)一直是整個系統中相當重要的環節,而在近期深層學習(deep learning)的發展之下,從自然語言處理(natural language processing)領域發展起來的自監督式學習(self-supervised learning)已經在該領域中獲得廣泛的成果,類似的方式也隨之開始被許多人嘗試用於語音訊號的表徵學習(representation learning),並成為一個新的特徵抽取方式,應用於各種語音處理的任務上,期望可以達成比傳統特徵工程(feature engineering)更好的表現。

        然而,雖然主要都是為了處理人類的語言,自然語言處理的文字和語音處理的語音訊號,在資料量和形式都有著巨大的差異,因此即便使用相似的訓練方式,作為離散(discrete)符號的文字與連續(continuous)訊號的語音仍可能造成訓練後得到的學習表徵差異甚遠。

        自然語言處理中的向量表徵常以學習語意(semantic features)為號召,可以讓模型學習相似的單字之間共同出現的頻率等特徵。但在語音處理中,複雜且變化多端的訊號卻可能需要先行抽取語音中的聲學特徵(acoustic features)辨別單字,才能進一步處理單字頻率甚至語意資訊等語言特徵(linguistic features)。另外,從資料處理的效率來看,同樣的一句話以語音表徵時所需要的資料長度本身相差數十倍,因此如何讓語音的自監督學習模型能夠更好的借鑒自然語言處理的學習模式,依然是一個相當值得探討的研究方向。

        為此,本論文旨在分析現有的各種語音表徵是否能夠從語音資料中抽取出類似文字模型的語意特徵,並進一步藉由嘗試對語音表徵進行分群、離散化等操作,使得原先較為複雜的語音表徵可以更加類似於文字模型的學習模式,以期望能夠學到更接近人類語意的向量表徵,並在語音的下游任務——特別是與內容相關的語音辨識和語音翻譯上獲得幫助。


%%%%%%%%%%%%%%%%%%%%%%%%%%%%%%
