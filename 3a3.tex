\subsection{基於語音學分類的分析}

        \begin{table}[!htbp]
            \centering
            \begin{subtable}[t]{\textwidth}
                \centering
                \begin{tabular}{cccccc}
                                & 語音類別標籤純度 & 分群純度 & 語音類別標籤熵 & 離散單元熵 &     NMI \\
                    HuBERT      &           0.7466 &   0.1422 &         1.7530 &     3.8681 &  0.5742 \\   %% h  1.0065
                    wav2vec 2.0 &           0.6913 &   0.1570 &         1.7530 &     3.8215 &  0.4682 \\   %% w  0.8208
                    CPC         &           0.7418 &   0.1953 &         1.7530 &     3.7918 &  0.5644 \\   %% c  0.9894
                    LogMel      &           0.5980 &   0.0953 &         1.7530 &     3.8630 &  0.3403 \\   %% l  0.5966
                \end{tabular}
                \caption{群數 = 50}
                \label{tab:ch3-clu050}
            \end{subtable}        

            \vspace{0.5cm}        

            \begin{subtable}[t]{\textwidth}
                \centering
                \begin{tabular}{cccccc}
                                & 語音類別標籤純度 & 分群純度 & 語音類別標籤熵 & 離散單元熵 &     NMI \\
                    HuBERT      &           0.7804 &   0.0856 &         1.7530 &     4.5704 &  0.6148 \\   %% h  1.0778
                    wav2vec 2.0 &           0.7219 &   0.0889 &         1.7530 &     4.5284 &  0.5252 \\   %% w  0.9207
                    CPC         &           0.7790 &   0.0997 &         1.7530 &     4.5034 &  0.6046 \\   %% c  1.0599
                    LogMel      &           0.6032 &   0.0567 &         1.7530 &     4.5591 &  0.3512 \\   %% l  0.6157
                \end{tabular}
                \caption{群數 = 100}
                \label{tab:ch3-clu100}
            \end{subtable}        

            \vspace{0.5cm}        

            \begin{subtable}[t]{\textwidth}
                \centering
                \begin{tabular}{cccccc}
                                & 語音類別標籤純度 & 分群純度 & 語音類別標籤熵 & 離散單元熵 &     NMI \\
                    HuBERT      &           0.8004 &   0.0464 &         1.7530 &     5.2681 &  0.6563 \\   %% h  1.1504
                    wav2vec 2.0 &           0.7490 &   0.0527 &         1.7530 &     5.2173 &  0.5671 \\   %% w  0.9941
                    CPC         &           0.7947 &   0.0644 &         1.7530 &     5.1885 &  0.6345 \\   %% c  1.1123
                    LogMel      &           0.6107 &   0.0335 &         1.7530 &     5.2322 &  0.3652 \\   %% l  0.6401
                \end{tabular}
                \caption{群數 = 200}
                \label{tab:ch3-clu200}
            \end{subtable}        

            \caption{不同群數在四種基石模型按照語音學類別的分析數據}
            \label{tab:single-cluster-phonetype-results}
        \end{table}
          根據表 \ref{tab:single-cluster-results} 中,在語音類別標籤的趨勢同樣在語音學類別中也能看得出來。然而,由於標籤數量較語音類別標籤少很多,因此語音標籤的純度相較音位會較高。
