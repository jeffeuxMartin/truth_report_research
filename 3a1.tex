\section{語音學分類(Phone Type)}

\subsection{簡介}

  除了單一音位本身的特性以外,由於音位本身彼此不是完全獨立的,而是彼此之間就存在相似的特徵,可以分成幾個組別。因此,依照 \mycite{tanghao等三篇} 的分組方式,對英語的音位進行分類並合併比對數據,看看這些離散單元本身是否有擷取到相似的發聲特徵,而不單純只是把音位分成約 50 類完全獨立的標籤。

% (基於語音表徵本身就是 acoustic sisgnals 來的,應該 by nature 要可以對語音特徵分組吧?)

        英語中的音位分為元音(Vowel)與輔音(Consonant)兩大類別,其中又可依照發音的共同特性一共分成七個類別。

\paragraph{元音}

% 母音在這邊為了簡單起見,會被分在一起?

根據發音的位置是否發生改變,英語的元音可分為:

\begin{itemize}
    \item 單元音(Monophthong)
    \item 雙元音(Diphthong)
\end{itemize}

兩大類別。

\paragraph{輔音}

而輔音按照發音的方式,可分為以下五類:

\begin{itemize}
    \item 爆破音 (Plosive)
    \item 擦音(Fricative)
    \item 塞擦音(Affircate)
    \item 鼻音(Nasal)
    \item 近音(Approximant)
\end{itemize}

\subsection{解釋意義}

\begin{itemize}
    \item 純度(Purity):換成以語音學的類別作為新的語音標籤後,有何變化(關聯性更強?)
    \item 熵(Entropy)(放直方圖解釋) \rightarrow 語音分類更明顯?
    \item 對齊(Alignment):是否減少分段資訊的保留(連續子音母音被合併?)
\end{itemize}
