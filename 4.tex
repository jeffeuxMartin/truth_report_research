\chapter{多個語音離散表徵組合與語音標記間的關係}

\section{動機}

基於單一語音單元表徵的限制,我們

放一些 rate 的東西


在现有的语音表征方法中,单一的离散单元虽然在许多任务中表现良好,但在处理长序列和时间分辨率方面存在局限性。Ren 等人的研究表明,许多相似的模式是以连续几个单元的形式出现的,这提示我们可以通过组合多个离散单元来更好地表示语音信号,从而提高模型在语音识别和语音翻译等任务中的性能 (\href{https://www.isca-archive.org/interspeech_2022/ren22_interspeech.html}{ISCA Archive}) (\href{https://deepai.org/publication/speech-pre-training-with-acoustic-piece}{DeepAI})。此外,SpeechTokenizer 的研究进一步展示了通过分层建模不同信息层次,能够在语音表示中捕捉到更丰富的语义和语音学信息 (\href{https://ar5iv.org/pdf/2308.16692}{ar5iv})。 


\section{相關研究}

\subsection{對語音離散表徵的分詞(tokenization)研究}

Ren 等人提出了使用声学片段(acoustic pieces)作为预训练的目标标签,通过对 HuBERT 代码进行句子片段化处理得到,能够显著改善自动语音识别(ASR)的性能 (\href{https://www.isca-archive.org/interspeech_2022/ren22_interspeech.html}{ISCA Archive})。Wu 等人介绍了 Wav2Seq,这是一种自监督方法,通过伪语言(pseudo languages)预训练编码器和解码器,显著提升了 ASR 和语音到文本翻译的性能 (\href{https://deepai.org/publication/speech-pre-training-with-acoustic-piece}{DeepAI})。

BEATs 提出了音频预训练方法,通过声学分片(acoustic tokenizers)学习音频离散表示,用于各种音频分类任务,表现出色 (\href{https://ar5iv.org/abs/2212.09058}{ar5iv})。

SpeechTokenizer 利用层级建模和语义蒸馏技术,通过多层残差量化器(RVQ)捕捉不同层次的信息,从而在语音表示中实现更高的语义和语音学特征保真度 (\href{https://ar5iv.org/pdf/2308.16692}{ar5iv})。





\section{分詞方法}


在文字語言模型上常用的分詞方法是 BPE,這種分詞方法比較於原本的 word 有幾個好處:

1 可以


 在文本处理中,常用的分词方法是 Byte Pair Encoding (BPE)。这种方法可以减少词汇表大小,提高模型的泛化能力,并能更好地处理罕见词和新词 (\href{https://deepai.org/publication/speech-pre-training-with-acoustic-piece}{DeepAI})。在语音处理中,类似的方法也可以用于离散语音单元的分词处理。例如,Ren 等人的方法通过句子片段化处理将高频代码模式合并为声学片段,从而有效地将输入音频与自然语言桥接起来 (\href{https://www.isca-archive.org/interspeech_2022/ren22_interspeech.html}{ISCA Archive}) (\href{https://deepai.org/publication/speech-pre-training-with-acoustic-piece}{DeepAI})。 


\paragraph{分词方法}

在文字语言模型上常用的分词方法是 Byte Pair Encoding (BPE)。这种分词方法相较于原本的单词表征有几个好处:

\begin{enumerate}
    \item \textbf{减少词汇表的大小}:提高模型的泛化能力。
    \item \textbf{处理罕见词和新词}:通过子词单元可以更好地处理罕见词和新词。
    \item \textbf{减少数据冗余}:提升训练和推理的效率。
\end{enumerate}
在语音处理中,类似的方法可以用于离散语音单元的分词处理。例如,Ren 等人的方法通过句子片段化处理(sentence piece)将高频代码模式合并为一个声学片段,从而有效地将输入音频与自然语言桥接起来 (\href{https://paperswithcode.com/paper/speech-pre-training-with-acoustic-piece}{Papers with Code})。

 


%%%%%%%%%%%%%%%%%%%
 

\section{衡量方式}

\subsection{字符(token)與音位之間的關係}




为了衡量离散单元和音位之间的关系,我们可以使用以下几个指标:

\begin{enumerate}
    \item \textbf{音位长条图(Phoneme Bar Chart)}:对单元进行频率统计,以了解每个单元在不同音位上的分布情况。
    \item \textbf{纯度(Purity)}:计算每个 cluster 单元内的音位纯度,以衡量单元对特定音位的代表性。 

% \textbackslash{}Epz(j)[py∣z(y∗(j)∣j)]\textbackslash{}E\_{p\_z(j)\}\left[p\_{y|z\}(y\^*(j)|j) \textbackslash{}right]\textbackslash{}Epz(j)[py∣z(y∗(j)∣j)] 

以及每个音位对 cluster 的纯度,衡量音位在 cluster 中的一致性。
% \textbackslash{}Epy(i)[pz∣y(z∗(i)∣i)]\textbackslash{}E\_{p\_y(i)\}\left[p\_{z|y\}(z\^*(i)|i) \textbackslash{}right]\textbackslash{}Epy(i)[pz∣y(z∗(i)∣i)]

    \item \textbf{熵(Entropy)和相互信息(Mutual Information, MI)}:从信息论角度探讨单元提供了多少背后音位的信息。 

% I(y;z)H(y)=∑i∑jpyz(i,j)log⁡pyz(i,j)py(i)pz(j)∑ipy(i)log⁡py(i)\textbackslash{}frac\{I(y;z)\}{H(y)\}=\textbackslash{}cfrac\{\sum\_i \textbackslash{}sum\_j p\_{yz\}(i, j) \textbackslash{}log \textbackslash{}cfrac\{p\_{yz\}(i, j)\}{p\_y(i)p\_z(j)\}}\{\sum\_i p\_y(i) \textbackslash{}log p\_y(i)\}H(y)I(y;z)=∑ipy(i)logpy(i)∑i∑jpyz(i,j)logpy(i)pz(j)pyz(i,j) \textbackslash{}begin\{align\} \textbackslash{}frac\{I(y;z)\}{H(y)\}&=\textbackslash{}cfrac\{\sum\_i \textbackslash{}sum\_j p\_{yz\}(i, j) \textbackslash{}log \textbackslash{}cfrac\{p\_{yz\}(i, j)\}{p\_y(i)p\_z(j)\}}\{\sum\_i p\_y(i) \textbackslash{}log p\_y(i)\} \textbackslash{} \&=\textbackslash{}frac\{H(y)-H(y|z)\}{H(y)\} \textbackslash{} \&=1-\textbackslash{}frac\{H(y|z)\}{H(y)\} \textbackslash{}end\{align\}
\end{enumerate}

\subsection{壓縮比率}



\subparagraph{字符(token)与音位之间的关系}

为了衡量离散单元和音位之间的关系,我们可以使用以下几个指标:

\begin{enumerate}
    \item \textbf{音位长条图(Phoneme Bar Chart)}:对单元进行频率统计,以了解每个单元在不同音位上的分布情况。
    \item \textbf{纯度(Purity)}:计算每个 cluster 单元内的音位纯度,以衡量单元对特定音位的代表性。
    \item \textbf{熵(Entropy)和相互信息(Mutual Information, MI)}:从信息论角度探讨单元提供了多少背后音位的信息。
\end{enumerate}

\subparagraph{压缩比率}

通过对离散单元进行分词和压缩,可以显著减少序列长度,从而提高训练和推理的效率。我们可以通过计算压缩比率来衡量这种方法的有效性。

 


\section{分析結果}


\subsection{基於各自音位的分析}

在此部分,我们将展示基于音位的频率统计结果和纯度分析结果。通过这些分析,我们可以了解离散单元在捕捉音位特征方面的表现 (\href{https://www.isca-archive.org/interspeech_2022/ren22_interspeech.html}{ISCA Archive}) (\href{https://deepai.org/publication/speech-pre-training-with-acoustic-piece}{DeepAI})。

\subsection{基於語音學分類的分析}


在此部分,我们将展示基于语音学分类的分析结果。这包括对离散单元在不同语音学类别(如元音、辅音)中的表现分析,以评估其在不同语音特征下的表现 (\href{https://www.isca-archive.org/interspeech_2022/ren22_interspeech.html}{ISCA Archive}) (\href{https://deepai.org/publication/speech-pre-training-with-acoustic-piece}{DeepAI})。



 
\section{應用在語音任務的實驗}

\subsection{語音辨識}

\subsubsection{實驗設定與資料集}

\subsubsection{實驗結果與其和分析數據間的關係}



\section{本章總結}

总结本章的内容,回顾离散单元在多个语音表征组合中的应用和优势,并展望未来的研究方向。
