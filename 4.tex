% !TeX root = ../thesis.tex

\chapter{多個語音離散表徵組合與語音標記間的關係}

\section{動機}

基於單一語音單元表徵的限制,我們

放一些 rate 的東西



\section{相關研究}

\subsection{對語音離散表徵的分詞(tokenization)研究}



\section{分詞方法}

在文字語言模型上常用的分詞方法是 BPE,這種分詞方法比較於原本的 word 有幾個好處:

1 可以



\section{衡量方式}

\subsection{字符(token)與音位之間的關係}

\subsection{壓縮比率}


\section{分析結果}

\subsection{基於各自音位的分析}

\subsection{基於語音學分類的分析}



\newpage


\section{應用在語音任務的實驗}

\subsection{語音辨識}

\subsubsection{實驗設定與資料集}

\subsubsection{實驗結果與其和分析數據間的關係}



\section{本章總結}
