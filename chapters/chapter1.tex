% !TeX root = ../thesis.tex

\chapter{導論}

\section{研究動機}

  語言是人與人之間溝通的最主要手段,而語音又是語言的一大形式。人類約有七千種語言,然而也許不是每個語言都有文字,但一定有其語音,因此發展語音科技是便利人們生活的一大必然。

  真的。really.

  隨著深層學習(deep learning)的崛起,自然語言處理和語音科技的進展快速,也有了許多幾乎可以匹敵人類的大模型,改變了人們生活的方方面面。

  然而,即便語音的技術已經相當成熟,然而語音技術開始與過往的人類對於語言學、語音的理解開始有了距離。似乎在為了讓機器可以擁有良好表現的同時,理解模型如何運作似乎是被犧牲的必然。但人們在追求更好的模型表現的同時,有一群人開始注意到語音處理模型是否有可能抓取(capture)到人類語音中特有的、區別於一般音訊的特徵,並嘗試使用過往用來研究、歸類人類語音的方式,結合機器學習與統計學去解釋為什麼,並期望可以比較甚至改善機器模型在進行語音處理時的表現,不僅僅只是使用資料集本身的分數,而有更多更多元、更穩健(robust)的衡量標準。由於離散表徵在當今語音模型與語音處理技術已經愈來愈具備重要性,因此探討與分析為什麼語音離散表徵可以幫助下游任務的背後成因是相當重要的研究方向,其中一個驗證離散表徵能夠幫助模型處理語音訊號的方式,便是驗證其與音位(phoneme)之間的對應關係。 


\section{研究方向}
  
\section{主要貢獻}  % 有什麼

結果我們發現,藉由觀察這些 unit 並嘗試藉由分詞演算法找出更 high-level 的單位之後,我們觀察到這些機器學習 figure out 的「偽標記」一定程度上的符合了人類音素的特性,因此可以當成某種類似拼音文字的存在。當然這跟人類真正使用的拼音文字仍有距離,
% // 難道可以探討 text 是不是一種 speech tokenization 過程嗎?

因此雖然無法直接當成 exact 的文字使用,一些人類的標注還是需要的,不過透過 ML 我們已經可以盡量更有效的利用珍貴的標注資料,幫助那些尚不容易取得文字的語言發展語音語言科技,以協助保存他們的語言。

% // 問題是,有些語音任務並不需要抽內容嗎?

\section{章節安排}

  由於本論文是以剖析既有的語音離散表徵為主軸,因此就
相關研究方面需要從各角度入手,單獨成一章節。接著我們會
從單一的離散單元,以及將單元視為像文字的字符
(character)並進行分詞演算法兩種對語音離散單元處理的
層次分別成章進行分析,最後將這些表徵嘗試做在語音的任務
上,以驗證其具有一定的語音表徵能力,且能保留語音學的特
徵。
