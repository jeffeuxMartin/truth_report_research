% !TeX root = ../../thesis.tex
\chapter{背景知識}

\section{深層類神經網路(deep neural network)}

深層類神經網路(Deep neural network) 是麥氏(McCulloch)在 1943 年提出 \cite{mcculloch1943logical} 仿生數學模型,旨在模擬生物神經系統的連結。

深層類神經網路是一個取法自生物神經連結的數學模型,其在計算認知神經科學中以連結派(connectionism)為主要代表,後在電腦科學與機器學習中有不同結構的進展。在此之後,基於其彈性與平行化的能力,能在 GPU 上面很有效率的進行運算並達到前所未有的效能,因此現在已經成為人工智慧發展的主流。

基於深層類神經網路的神經架構有 CNN、RNN、Transformer 等等,由於這些架構在語音與文字處理上都已經被廣泛使用,因此在下面分別介紹:


\subsection{卷積式(convolutional)類神經網路}

卷積式類神經網路一開始是在 \\cite{} 中提出,主要是鑑於影像中的局部性(locality),讓 NN 可以在

。

在語音中,因為

語音訊號的資訊是被呈現在時間維度上
,
因此
通常使用

一維

的
卷積式類神經網路
,
以捕捉
時間維度上的局部性特徵
,例如本研究特別探討的 phoneme、morpheme 等等。

% \subsection{遞迴式(recurrent)類神經網路}



\subsection{
    序列至序列(sequence-to-sequence)模型}

    


\subsection{專注(attention)機制}    

\subsection{轉換器(Transformer)}



\section{表徵(representation)學習}


\subsection{文字的語意表徵}

\subsection{語音特徵與表徵}

\section{語音基石模型與自監督式學習}

\subsection{自監督式學習}

\subsection{語音基石模型}

\subsection{離散單元}

\section{本章總結}
