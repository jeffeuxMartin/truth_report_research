% !TeX root = ../../../thesis.tex

\section{研究方向}

在近年,已經有不少相關的研究開始嘗試往將離散單元(discrete unit)作為除了連續表徵(continuous representation)以外,可以編碼(encode)語音訊號的另外一種方式。離散表徵(discrete representation)跟連續表徵相比,具有資訊更濃縮(位元率(bit rate)更低)因此更好儲存、處理與傳輸,以及形式上更像文字的特性。

儘管離散表徵在語音社群(community)中常被當成一種類似文字的存在,另外有一些文獻則是將其當成連續表徵的精簡表示法。

然而

因此,我們藉由分析各種離散單元和人類理解語音最直接的處理層次:音位(phoneme)之間的關係,並將兩者進行各種統計上的序列比較,可以作為訓練大型語音基石模型(foundation model)的分詞(tokenization)基礎,選定最適合的表徵最為系統的輸入符記(token)。

如 \cite{wu2023wav2seq} 等作品便是首先嘗試將離散單元進行 tokenization 後做進一步處理的,其後的
