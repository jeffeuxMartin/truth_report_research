% !TeX root = ../../../thesis.tex

\section{研究方向}  % 做什麼

於是我們對這些離散單元進行了單一的分析
另外,因為人類的音素往往是超過一個音框的單位
因此推測離散單元單一本身會抓取到的是更加細緻的特徵

那為了找出更加符合人類文字認知的表示符號
因此我們採用文字那邊常用的分詞器
// 可能要紀錄 rate 來證明這件事了
嘗試看看可否找出更接近人們認知信息率的符號,以其這種機器抓出來的 fine structure 可以更往人類的文字靠近

具體方法是,針對這些抓出來的 unit 跟 piece, 我們去算 purtiy,pmni, segmentation 這些指標,看看他們是不是真的跟人類 top down 標註出來的的 phoneme label 有統計上的一致性。
最後我們嘗試做在 PR 或 ASR 上,驗證這個一致性不是僅僅在統計數據上顯現,也是可以在應用上得到幫助的。