\section{研究動機}

  首先,語言是人溝通很重要的,就對外溝通最主要的手段,所以科技的發展,其中為了協助人與人之間溝通,語言的科技是一定要發展的,就不可能漏掉語言這一塊,那過往在還沒有那麼多資源的時候,都是由就是依賴機器學習,跟語言學家的努力的技術來達成這個目的,就我過往知識這一點非常重要,然而隨著近年就是硬體的崛起,那這些機器學習,就開始,可以利用大量未標註,就是相對容易取得,而不需要再像過往那樣費時費工的,依賴人類的標註,然而在機器往這個方向發展的同時,那個解釋性以及過往語言知識就逐漸的脫節,那為什麼脫節需要被解決呢?因為人們才能更好地去解決問題,人們才能更好地使用這些模型,以及怎麼樣去改善它,就不會單純認為就是它就是個黑箱子,然後在發展機器的就是發展這些語言科技的時候,能夠更加貼近這種技術,然後就語音這一塊,為什麼語音會多於文字,一來是因為文字它比較好訓練,現在已經有許多的機實模型,反而在語音的訊號本身,就比文字複雜非常多,還有很多語者啦情緒啦等等的消息包含在裡面,因此就利用率上面來講,它沒有像文字那麼好處理,因此要去處理語音訊號是非常重要的,而且一來這是人類最主要的語言的使用方式,文字還需要從小開始用來解釋的方式,慢慢學習,但是語音基本上你只要處在人類社會,跟其他人有所交流,幾乎人類都會使用語言,更何況世界上7000多種語言當中,有許多是沒有文字的,那為了讓這些講其他語言還沒有辦法,就是反過來你可以想說,一在各種語言之間翻譯需要成本,那需要當地的熟悉當地文化的人才能怎麼樣怎麼樣,那對於這些講少數語言的人,他們也會花時間,可能還需要去學習母語以外的語言,那如果說我們今天可以把語言科技本身,藉由語音這個更好的更貼近於人類本身習慣的媒介,去直接讓他們可以運用他們最熟悉的母語表達,那對於語言本身以及文化等各個層面,一定是個更大的進步,好,那我現在就想說,我們在這個世界上,我們要知道要去能夠在機器學習做語音的前提之下,發展科解適性以及能夠讓人類知道這個前提之下,那離散特徵本身,它是近期在做機器學習這一塊當中很重要的技術,因為它能夠在不需要人類各自labor去標註那些文字的同時,達到一定程度上的去處理,這是一個很大的進展,然而它本身究竟有多麼的像文字,以及它可不可以被拿來像是大型語言模型一樣,拿來當成文字訓練呢,近年是有這麼一些嘗試,那例如GSM就是一個標竿,但是它本身,它本身的效率就是還沒有像文字那麼好,那為了去思考這個問題,我們可以倒回去從,Unit它最基本最基本的語音的第一層,也就是Phonology去討論這件事,看它跟聲學特徵跟語音到底有多像,從這個方向我們可以進而去改善說,也許我們可以去用不同的方式使用Unit,來進一步推進讓我們的Unwritten Language這些技術,能夠更好的跟,就是等於說進而發揮類似文字的效果,然後同時也能促進我們去知道,機器是怎麼樣去理解這些語音訊號,它可能看的是哪些部分。

\section{研究方向}

  在過往他們做Super Secret Unit的這些representation的當中,Purity的基本指標就是去elaborate說他們今天discover出來的那個unit本身,與Foening之間的一致性有多高,所以一定會說Purity以及PNMI這兩個指標,然後呢,他們我們就先對基本的一些unit,尤其是在Texas這個里程碑上面去做這些分析,然後接下來我們會嘗試想說就是有一個假設是,基於有文獻以及從語言學上面看到的一些現象,我們可以發掘,比如說,或許,或許,Unit itself does not represent the phonological structure at all,only,就是你會需要一個Sequence,也許Deduplication本身很重要,所以我們會比較說,如果借用文字那邊VP,就有點像從Button-up裡面去,從更fine structure的信號角度去,組成一個人類從Top-down認知的Foening這個角色之間,我們可不可以借用文字的organization演算法,去達成discovery這些東西的目的,然後一樣去分析這兩個,他們的那些資訊的角度,就是把這個單位當成新的Acoustic unit,就是Acoustic unit discovery他們在做的事情,那這就是一個簡單的分析的方式,最後我們會把它測試在Foening recognition,畢竟我們discovery如果是Foening,它應該有一個準確率,但在應用的角度上我們會做Speech recognition這樣子,因為這就能直接反應到,它有沒有抽到裡面的content,OK

