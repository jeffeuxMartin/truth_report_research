% this file is encoded in utf-8
% v3.0 (Jun. 11, 2019)

\documentclass[12pt, a4paper]{ntuthesis}

% 除非校方修改了論文格式 (margins, header, footer, 浮水印, 中文數字之章別)
% 或者需要增加所用的 LaTeX 套件,
% 或者要改預設中文字型、編碼
% 否則毋須修改本檔內容

%%%%%%%%%%%%%%%%%%%%%%%%%%%%%%
\usepackage{mathtools} % 各式 AMS 數學功能
\usepackage{amssymb} % 各式 AMS 數學符號
\usepackage{mathrsfs} %草寫體數學符號,在數學模式裡用 \mathscr{E} 得草寫 E
\usepackage{bm}
\usepackage{caption}
\usepackage{subcaption}
\usepackage{tabularx}
\usepackage{url}
\usepackage[usenames,dvipsnames]{color}
%\usepackage[square, comma, numbers, sort&compress]{natbib}
\usepackage[
    colorlinks, 
    linkcolor = black, 
    citecolor = red,    % black
    urlcolor  = blue,   % black
    unicode
]{hyperref}

\usepackage{siunitx}
\usepackage{tikz}
\usetikzlibrary{arrows.meta,positioning,calc,chains,scopes}
\usetikzlibrary{shapes.geometric,angles}
\usetikzlibrary{backgrounds,fit}

\usepackage{multirow,multicol,rotating}
% 插圖套件 graphicx
\usepackage{graphicx}
% font, margin, linestretch 是設在 ntuthesis.sty 裡
\usepackage{ntuthesis}


%%%%%%%%% listings / algorithm %%%%%%%%%%
\usepackage{listings} % 程式列表套件
% hyperref跟algorithm衝突,hyperref必須放在algorithm前面
\usepackage{algorithm}
\usepackage{algorithmic}

%
% listing setting
\lstset{breaklines=true,% 過長的程式行可斷行
extendedchars=false,% 中文處理不需要 extendedchars
texcl=true,% 中文註解需要有 TeX 處理過的 comment line, 所以設成 true
comment=[l]\%\%,% 以雙「百分號」做為程式中文註解的起頭標記,配合 MATLAB
basicstyle=\small,% 小號字體, 約 10 pt 大小
commentstyle=\upshape,% 預設是斜體字,會影響註解裏的英文,改用正體
%language=Octave % 會將一些 octave 指令以粗體顯示
}
%%%%%%%%%%%%%%%%%%%%%%%%%%%%%%%%%%%%%%%%%

%%%%%%%%% fancyhdr %%%%%%%%%%
% 增強功能型頁楣 / 頁腳套件
\usepackage{fancyhdr}  % 借用此套件來擺放浮水印 
% (佔用了 central header)
% 不需要浮水印的使用者仍可利用此套件,產生所需的 header, footer
%
% 啟動 fancy header/footer 套件
\pagestyle{fancy}
\fancyhead{}  % reset left, central, right header to empty
\fancyfoot[C]{\thepage} %中間 footer 擺放頁碼
\renewcommand{\headrulewidth}{0pt} % header 的直線; 0pt 則無線

% 如果不需要任何浮水印,則請把下列介於 >>> 與 <<< 之間
% 的文字行關掉 (行首加上百分號)
%% 浮水印 >>> 
% 請用 Adobe Acrobat 加入
% 也會需要加入 doi 的浮水印
%\input{watermark}
%% <<< 浮水印

% 如需額外的頁楣 (header) 或 footer,請在 headerfooter.tex 裡依例修改
% 它的預設內容是都關掉,可依需要打開
\input{headerfooter}
%%%%%%%%%%%%%%%%%%%%%%%%%%%%%

% 常見的 notation 指令
% 請修改 notations.tex 加入自己需要的 notations
% %%%% Common Symbol %%%%
\newcommand*{\Nb}{\mathbb{N}}
\newcommand*{\Zb}{\mathbb{Z}}
\newcommand*{\Qb}{\mathbb{Q}}
\newcommand*{\Rb}{\mathbb{R}}
\newcommand*{\Cb}{\mathbb{C}}
\newcommand*{\Eb}{\mathbb{E}}

\newcommand*{\Ac}{\mathcal{A}}
\newcommand*{\Dc}{\mathcal{D}}
\newcommand*{\Oc}{\mathcal{O}}
\newcommand*{\Tc}{\mathcal{T}}
\newcommand*{\Uc}{\mathcal{U}}
\newcommand*{\Vc}{\mathcal{V}}
\newcommand*{\Xc}{\mathcal{X}}
\newcommand*{\Yc}{\mathcal{Y}}
\newcommand*{\Zc}{\mathcal{Z}}
\newcommand*{\trans}{\mathsf{T}}

%%%% mapping Symbol %%%%
\newcommand*\bij{\lhook\joinrel\twoheadrightarrow}
\newcommand*\oneto{\hookrightarrow}
\newcommand*\onto{\twoheadrightarrow}
\newcommand*\isoto{\xrightarrow{\sim}}
\newcommand*\acts{\curvearrowright}
\newcommand*\revacts{\curvearrowleft}

%%%% set definition %%%%
% just to make sure it exists
\providecommand\given{}
% can be useful to refer to this outside \Set
\newcommand*\SetSymbol[1][]{%
  \nonscript\:#1\vert
  \allowbreak
  \nonscript\:
\mathopen{}}
\DeclarePairedDelimiterX\Set[1]\{\}{%
  \renewcommand\given{\SetSymbol[\delimsize]}
  \,#1\,
}

%%%% Probability and Statistics Notations %%%%
\DeclarePairedDelimiterXPP{\KL}[2]{D_\text{KL}}\lbrack\rbrack{}{{#1} \delimsize\Vert {#2}}
\DeclarePairedDelimiterXPP{\Prob}[1]{\Pb}\lbrace\rbrace{}{#1}
\DeclarePairedDelimiterXPP{\Ev}[1]{\Eb}\lbrack\rbrack{}{#1}
\DeclarePairedDelimiterXPP{\Evr}[2]{\Eb_{#1}}\lbrack\rbrack{}{#2}

%%%%%%%%symbol and function settings%%%%%%%%%
\DeclarePairedDelimiter{\abs}{\lvert}{\rvert}
\DeclarePairedDelimiter{\norm}{\lVert}{\rVert}
\DeclarePairedDelimiter{\inpd}{\langle}{\rangle} % inner product
\DeclarePairedDelimiter{\ceil}{\lceil}{\rceil}
\DeclarePairedDelimiter{\floor}{\lfloor}{\rfloor}
\DeclareMathOperator*{\argmin}{arg\,min}
\DeclareMathOperator*{\argmax}{arg\,max}
\newcommand*{\bv}[1]{\mathbf{#1}}

%% some function name
\DeclareMathOperator{\ELBO}{ELBO}
\DeclareMathOperator{\ReLU}{ReLU}



% Title Page
% 請修改 title_settings.tex 設定論文題目/指導教授等變數
\renewcommand{\enTitle}{Research of Discrete Speech Representations}  %英文標題
\renewcommand{\zhTitle}{語音離散化表徵}  %中文標題
\renewcommand{\authorZhName}{陳建成}  %作者中文姓名
\renewcommand{\authorEnName}{Chien-Cheng Chen}  %作者英文姓名
\renewcommand{\authorStudentID}{R09942097}  %作者學號
\renewcommand{\advisorZhName}{李琳山}  %指導教授中文姓名
\renewcommand{\advisorEnName}{Lin-shan Lee}  %指導教授英文姓名
\renewcommand{\zhCollegeName}{電機資訊學院}  %學院中文名稱
\renewcommand{\enCollegeName}{College of Electrical Engineering and Computer Science}  %學院英文名稱
\renewcommand{\zhDepartmentName}{電信工程學研究所}  %系所中文名稱
\renewcommand{\enDepartmentName}{Graduate Institute of Communication Engineering}  %系所英文名稱
\renewcommand{\rocYear}{一百一十三}  %中華民國紀年年份
\renewcommand{\zhMonth}{六}  %中文月份
\renewcommand{\enYear}{2024}  %公元紀年
\renewcommand{\enMonth}{June}  %英文月份
\renewcommand{\oralDate}{113 年 6 月 1 日}  %口試日期


% 載入中文名詞的定義:例如,Figure -->「圖」, Chapter -->「第 x 章」
\input{ntu_definitions}

% 如果不需要以中文數字一、二、三呈現章別,例如「第一章」
% 則請把下列介於 >>> 與 <<< 之間
% 的文字行關掉 (行首加上百分號), 會以「第 1 章」呈現
%% 中文數字章別 >>>
\input{ntu_chnum}
%% <<< 中文數字章別

%%%%%%%%%%%%%%%%%%%%%%%%%%%%%
%  end of preamble
%%%%%%%%%%%%%%%%%%%%%%%%%%%%%
%
\begin{document}
\CJKindent  %%% ZZZ %%%  段首內縮兩格

%%% 以下載入前頁、本文、後頁

% \NTUtitlepage  % 產生論文封面

\newpage
\setcounter{page}{1}
\pagenumbering{roman}

\NTUoralpage  % 產生口試委員會審定書

\mydoublespacing
\begin{acknowledgement} %誌謝
    兩年來的碩士生活,隨著這份論文的誕生而告了一個段落。回思過去兩年來的生活,有做出成果的喜悅、也有處處碰壁時的苦悶、也有跟同學共同奮鬥的記憶,自己在這兩年中實在成長了許多,而這些都要感謝實驗室的大家長:李琳山教授。教授在實驗室營造了自由研究的氛圍,讓實驗室的同學都能按自己喜好自由發展研究方向,並從旁關心協助同學的研究。我在這樣的氛圍下也受益許多,學習到了許多做研究與做人處事的方法。

    這些研究能夠順利完成要由衷地感謝我的家人,他們無論何時都支持我的決定,並且從旁給我協助,在我最忙碌而都很晚回家的那段時間,他們也是很包容我,並給予我支持。

    與實驗室的同學相處的這段時光將是碩士生活中最難忘的一段日子,我們無論是修課或是研究上都是彼此的好戰友,我從你們身上都學到很多;下一屆的你們將是實驗室下一代的主力,祝你們未來研究順利!

    最後要感謝我的朋友、同學、以及伙伴們,不論是平常一起吃飯聊天、有正事時的一起奮鬥、或一起出去玩,你們都是我平常生活上最大的支持!

\end{acknowledgement}

\begin{zhAbstract} % 中文摘要
    這是中文摘要
\end{zhAbstract}

\begin{enAbstract} % 英文摘要
    This is an English abstract.
\end{enAbstract}

{
%\zhKaiFont
\mysinglespacing\selectfont
\tableofcontents %目錄

\listoffigures  %圖目錄

\listoftables  %表目錄
\par
}

\newpage
\setcounter{page}{1}
\pagenumbering{arabic}
  % omitted as...: 
% <<<
% \NTUtitlepage  % 產生論文封面

% \newpage
\setcounter{page}{1}
\pagenumbering{roman}

% \NTUoralpage  % 產生口試委員會審定書

\mydoublespacing
%% \begin{acknowledgement} %誌謝
%%     兩年來的碩士生活,隨著這份論文的誕生而告了一個段落。回思過去兩年來的生活,有做出成果的喜悅、也有處處碰壁時的苦悶、也有跟同學共同奮鬥的記憶,自己在這兩年中實在成長了許多,而這些都要感謝實驗室的大家長:李琳山教授。教授在實驗室營造了自由研究的氛圍,讓實驗室的同學都能按自己喜好自由發展研究方向,並從旁關心協助同學的研究。我在這樣的氛圍下也受益許多,學習到了許多做研究與做人處事的方法。%% 

%%     這些研究能夠順利完成要由衷地感謝我的家人,他們無論何時都支持我的決定,並且從旁給我協助,在我最忙碌而都很晚回家的那段時間,他們也是很包容我,並給予我支持。%% 

%%     與實驗室的同學相處的這段時光將是碩士生活中最難忘的一段日子,我們無論是修課或是研究上都是彼此的好戰友,我從你們身上都學到很多;下一屆的你們將是實驗室下一代的主力,祝你們未來研究順利!%% 

%%     最後要感謝我的朋友、同學、以及伙伴們,不論是平常一起吃飯聊天、有正事時的一起奮鬥、或一起出去玩,你們都是我平常生活上最大的支持!%% 

%% \end{acknowledgement}%% 

%% \begin{zhAbstract} % 中文摘要
%%     這是中文摘要
%% \end{zhAbstract}%% 

%% \begin{enAbstract} % 英文摘要
%%     This is an English abstract.
%% \end{enAbstract}

%%%%%% 先藏起來
%%%%%%%% {
%%%%%%%% %\zhKaiFont
%%%%%%%% \mysinglespacing\selectfont
%%%%%%%% \tableofcontents %目錄
%%%%%%%% 
%%%%%%%% % \listoffigures  %圖目錄
%%%%%%%% 
%%%%%%%% % \listoftables  %表目錄
%%%%%%%% \par
%%%%%%%% }
%%%%%% <<<

\newpage
\setcounter{page}{1}
\pagenumbering{arabic}

% >>>
\input{1}
% 處理 units 的 Gumbel 那些

\chapter{背景知識}

\section{深層類神經網路}

\subsection{簡介}

  
深層類神經網路(Deep Neural Network,DNN)是由神經科學家麥氏(McCulloch)與皮氏(Pitts)於 1943 年提出 \cite{mcculloch_logical_1943} 的計算模型,靈感取自連結主義(Connectionism)的核心主張 --- 以模仿生物神經網路的連結方式模擬複雜的心智活動。






 
為模擬神經細胞處理訊號的過程,深層類神經網路最基本的單位稱為「神經元(Neuron)」,其本質為線性分類器。每個神經元接收的輸入數值 $x = (x_1, x_2, \cdots\cdots, x_N)$ 是一個 $N$ 維向量,每一維會被賦予一個權重(Weight) $w = (w_1, w_2, \cdots\cdots, w_N)$  ,加權後總和再加上偏差值 $b$,得到線性輸出值。為了模擬神經細胞的觸發過程,該分類器常被加上非線性的激發函數(Activation Function)$\sigma$ 的轉換,才得到最終輸出值 $y$。如圖 \ref{fig:single-neuron}所示,神經元的運算規則以下列數學式描述:
$$y = \sigma(w^T x + b)$$
常見的激發函數包含線性整流單元(Rectified Linear Unit,ReLU)、S 函數(Sigmoid Function)或雙曲正切函數(Hyperbolic Tangent Function,$\tanh$)等等。


\begin{figure}
    \centering
    \includegraphics[width=0.5\linewidth]{figures/neuron.drawio.png}
    \caption{神經元示意圖}
    \label{fig:single-neuron}
\end{figure}


結合數個神經元的運算,羅氏(Rosenblatt)於 1958 年 \cite{rosenblatt_perceptron_1958} 提出感知器(Perceptron)模型。根據通用近似定理(Universal Approximation Theorem)\cite{funahashi_approximate_1989} ,感知器理論上可逼近任意函數。然而,後續研究發現單層的感知器具有如「線性不可分」\footnote{例如無法貼合異或(Exclusive OR,XOR)運算等函數} 等先天限制,使其曾經一度不被看好。

為了突破該缺陷,人們嘗試在輸入與輸出層之間增加「隱藏層(Hidden Layer)」,成為「多層感知器(Multilayer Perceptron,MLP)」,如圖 \ref{fig:mlp} 所示。藉助隱藏層的幫助,多層感知器可對輸入進行多次非線性轉換,大大拓展了模型的適用範圍。此模型是透過「加深隱藏層」得來,現今為人們熟知的「深層類神經網路(Deep Neural Network)」即由此得名。

\begin{figure}
    \centering
    \includegraphics[width=0.8\linewidth]{figures/nnout.png}
    \caption{多層感知器/深層類神經網路示意圖}
    \label{fig:mlp}
\end{figure}


藉助深層類神經網路的彈性,我們可以透過⼤量訓練資料來訓練模型,藉此逼近應⽤任務中欲近似的函數 $f$,該函數蘊藏在資料集 $\mathcal{D} = \{(x_i, y_i)\}_{i=1}^N$ 中,其中每個資料點 $(x_i, y_i)$ 為輸入與輸出間的配對,即對於 $N$ 個資料點都有 $$y_i = f(x_i) \ \  \forall i \in \{1, \cdots\cdots, N\}$$之關係。
為了使這個函數更加逼近目標函數 $f$,
類神經網路會構建一個逼近中的函數 $f_{\theta_t}(\cdot)$ 。
透過不停的迭代,
模型對資料集 $\mathcal{D}$ 的每一筆資料 $x$ 給出預測 $f_{\theta_t}(x)$ 。
透過某個減損函數(Loss Function)$\mathcal{L}$ 計算出誤差(Error),
此誤差對參數 $\theta_t$ 求出梯度(Gradient)後將指示模型更新的方向,
以此乘上學習率(Learning Rate)$\eta$ 後從參數  $\theta_t$ 減去,便能對整個模型進行更新,使之更有機會接近目標函數 $f$。
由於此過程是依照梯度使得函數 $\mathcal{L}$ 逐步降低,以此獲名「梯度下降法(Gradient Descent)」,其公式如下:
$$\theta_{t+1} \leftarrow \theta_{t} - \eta \nabla_\theta\mathcal{L}(\mathcal{D}, f_{\theta_t}(\cdot))$$
其中,$t$ 為當前的迭代數,$\theta_t$ 為當前模型參數,$\theta_{t+1}$ 為更新後的模型參數。

在此模型更新的過程中,減損函數承擔著指引模型逼近的角色,因此根據應用的任務不同,常見的減損函數包括
\begin{itemize}
    \item 均方誤差(Mean Squared Error,MSE):一般用於迴歸(Regression)問題,直接計算兩數值之間的差距的平方和
    $$\mathcal{L}_{\text{MSE}}(y_i, \hat{y}_i) = \frac{1}{N} \sum_{i=1}^{N} (y_i - \hat{y}_i)^2$$

    \item 交叉熵(Cross-entropy,CE):一般用於分類(Classification)問題,著重計算兩個機率分佈之間的差異
    $$\mathcal{L}_{\text{CE}}(y_i, \hat{y}_i) = -  \sum_{i=1}^{N} \left[ y_i \log(\hat{y}_i) + (1 - y_i) \log(1 - \hat{y}_i) \right]$$
\end{itemize}

透過上述的訓練方式可以得知,類神經網路的訓練需要相當龐大且複雜的運算過程,因此剛提出時仍舊難以應用於現實應用中。

為了提高函數貼合的效率,魯氏(Rumelhart)與辛氏(Hinton)等人 \cite{rumelhart_learning_1986, rumelhart_learning_1987} 提出了反向傳播(Backpropagation)演算法,旨在將上述的更新過程,藉助鏈鎖率(Chain Rule)的幫助,由隱藏層逐層反向傳播至輸入層,對整個類神經網路進行修正。

反向傳播演算法的設計,正好能配合圖形處理器(Graphics Processing Unit,GPU)等硬體裝置的優勢,以平行運算能力加速函數貼合(Fit)的效率。由此開始,這種透過深層類神經網路,從大量資料集中發掘函數關係的機器學習演算法,被稱為深層學習(Deep Learning)。 類神經網路在各個領域的泛化能力(Generalizability)已經得到前所未有的效能,包含電腦視覺、語音處理和自然語言處理,因此深層學習在近年成為人工智慧發展的主流。

然而,根據資料特性的不同,並不是所有的資料都適用簡單的「輸入與輸出配對」的模式。研究者根據任務需求,發展出了不同架構的類神經網路以適應資料特性  。
前述最基本的深層類神經網路,由於資料是直接由輸入層,通過逐層的矩陣運算得到輸出,因此被稱之為「前饋式類神經網路(Feed Forward Network,FFN)」。

藉由調整各神經元之間的連接關係,發展出卷積式(Convolutional)、遞迴式(Recurrent)與轉換器(Transformer)類神經網路等架構變體,以適應如影像、語音和文字等不同型態的資料。這些架構在語音與文字處理被普遍使用,接下來將逐一分別介紹:

\subsection{卷積式類神經網路}
  
卷積式類神經網路(Convolutional Neural Network,CNN)為 1998 年由楊氏(Yann LeCun) \cite{lecun_gradient-based_1998} 提出,旨在以訊號處理的卷積(Convolution)運算,模擬生物的視覺皮質感知 \cite{hubel_receptive_1959} 。

\begin{figure}
    \centering
    \includegraphics[width=0.9\linewidth]{figures/cnnnew.png}
    \caption{卷積式類神經網路示意圖,取自李宏毅教授的課程投影片}
    \label{fig:cnn}
\end{figure}

如圖 \ref{fig:cnn} 所示,卷積式類神經網路透過核心(Kernel),對輸入的資料 --- 如圖中的二維矩陣 --- 進行卷積運算,獲得該輸入的特徵圖(Feature Map)。核心帶來的移動不變性(Shift-invariance)非常適用於捕捉二維影像中的局部特徵,以作為類神經網路分辨資料的依據。

有別於影像處理中,資料多以二維矩陣表示像素 (Pixel)三原色的亮度數值,因此以二維的卷積運算為主;
由於語音時常處理時間軸之上的訊號,包含聲波波形(Waveform)、時頻譜(Spectrogram)或聲學特徵,因此一維的卷積式模型也時常出現,以模仿人耳聽覺對時變訊號的窗框(Window)的效應,進而觀察到語音中在不同解析度(Resolution)的資訊。


\subsection{遞迴式類神經網路與序列至序列模型}

\subsubsection{遞迴式類神經網路}
  
遞迴式類神經網路(Recurrent Neural Network,RNN)常用於處理隨時間變化的序列資料,特別是語音與文字等等,順序資訊相當關鍵的各種語言任務。

為了處理需要記憶和狀態的資料類型,
遞迴式類神經網路的輸出會重新接回輸入層,
使得前一個時間點(Timestep)的資料與內部狀態
會繼續影響後續的時間點。

常用的遞迴式類神經網路類型有長短期記憶(Long Short-term Memory,LSTM)\cite{hochreiter1997long} 和閘門循環單元(Gated Recurrent Unit,GRU)\cite{cho-etal-2014-properties} 等。

遞迴式類神經網路通常用在處理序列至序列的應用,例如語音辨識、語音合成或機器翻譯等和語言密切相關的任務中。

\subsubsection{序列至序列模型}
  
由於許多語言資料通常以兩個序列互相配對的形式呈現,因此專門處理這類資料的模型被稱為序列至序列模型(Sequence-to-sequence,Seq2seq)\cite{sutskever2014sequence}。此類模型的典型架構由編碼器(Encoder)和解碼器(Decoder)組成,旨在模擬輸入與輸出序列之間的變化與相依關係(Dependency)。

序列到序列模型一般有兩種模式:其一是每個時間點都生成一個輸出的向量,適用於輸入與輸出序列等長的任務,這種模式被稱為符記分類(Token Classification);但更常見的情況是,輸入與輸出序列的長度並不相同。處理後者的典型作法是讓編碼器將輸入序列依據時間,一步一步輸入編碼器,將序列編碼為內部表徵(Latent Representation)。完成編碼後,編碼器將最後一個時間點的表徵代表用以整個序列,稱為「語境向量(Context Vector)」。該向量接著被傳遞給解碼器,依序生成輸出序列。
\subsection{專注機制與轉換器類神經網路}

\subsubsection{專注機制}
  
由於遞迴式類神經網路需要處理整個序列的編碼和解碼資訊,對時間點距離較遠的輸入容易被遺忘,亦即難以處理長期相依性(Long-term Dependency)問題。為了解決這種困境,巴氏(Bahdanau)等人提出了「專注機制(Attention Mechanism)」\cite{bahdanau2014neural}。該機制讓解碼器將每個輸入序列的訊號都視作「部分的」語境向量,由對不同時間點的向量加權合計獲得,使得在生成輸出序列時能依據當時的需求從輸入序列中提取所需的訊息。專注機制的引入,使得序列至序列模型在處理如語音辨識、機器翻譯等任務時大大改善了效能。

\subsubsection{轉換器類神經網路}
  
儘管遞迴式類神經網路善於處理時序資料,但其難以平行化的架構限制了其在訓練和推理(Inference)時的效率。2017 年,瓦氏(Vaswani)等人 \cite{vaswani2017attention} 提出了完全由專注機制構成、不依賴遞迴運算的序列至序列模型,並稱之為「轉換器(Transformer)」,以解決機器翻譯等任務。

轉換器類神經網路一般包含編碼器和解碼器兩部分,均為多層架構。圖 \ref{fig:tfm_arch} 展示完整的轉換器架構圖,以下分別介紹其主要元件:

\paragraph{位置編碼(Positional Encoding)}

對於編碼器或解碼器的輸入序列,模型先對序列中不同位置的時間點進行編碼,取代遞迴式類神經網路逐步運算的過程,使其能在平行計算的同時考慮不同時間點的影響。編碼的函數可依照需求變換,如原始的轉換器採用三角函數進行位置編碼,而在語音模型中,有時也會採用卷積式網路以捕捉輸入的細微資訊。

經過位置編碼後,向量會通過每一個轉換器層(Transformer Layer),進行以多頭專注(Multi-head Attention)為主的一連串運算:

\paragraph{多頭專注}

轉換器層中的專注機制涉及三個輸入向量:詢向量(Query)$Q$、鑰向量(Key)$K$ 和值向量(Value)$V$。專注機制運算如下:
\[
\text{Attention}(Q, K, V) = \text{softmax}
\left(
\frac{QK^\top}{\sqrt{d_k}}
\right)
V
\]
其中 $\text{softmax}$ 為正規化指數函數,$d_k$ 為鑰向量 $K$ 的維度。這一運算首先通過鑰向量和詢向量的內積計算專注權重,而後為避免受維度過大影響而縮小為 $\sqrt{d_k}$ 分之一,最後通過正規化指數函數使得權重總和為 1 ,以此分配給值向量進行加權。

為應對多樣的輸入訊號,每個轉換器層具備多個獨立的專注機制,對三組輸入向量先進行各自不同的 $W^Q$、$W^K$、$W^V$ 線性轉換,稱為「多頭專注(Multi-head Attention)」。對於第 $i$ 個專注頭(Head)有
\[
\text{head}_i = \text{Attention}(QW^Q_i,KW^K_i,VW^V_i)
\]
最後,若有 $h$ 個專注頭,多頭專注模組會將多個頭的結果進行串接(Concatenate),經過線性轉換 $W^O$ 作為模組輸出
\[
\text{MultiHead}(Q, K, V) = \text{Concat}(\text{head}_1, \cdots\cdots, \text{head}_h) W^O
\]

\paragraph{其他層內運算}

每層轉換器層在經過多頭專注運算後,會依序進行以下三個步驟:

\begin{enumerate}
\item 與輸入向量透過殘差連接(Residual Connection)相加,隨後進行層正規化(Layer Normalization)以穩定訓練。
\item 將此結果通過一個簡單的前饋式類神經網路對向量做線性轉換。
\item 再將前饋網路的輸入與輸出再次計算殘差總和後,進行層正規化輸出。
\end{enumerate}

以上為轉換器被提出時的最原始模型,其後對殘差連接、層正規化的安排也存在各類變體。

\paragraph{跨專注機制(Cross-atttention)}

由於解碼器需要來自編碼器的輸入序列資訊幫助輸出,因此,原本在編碼器層中的自專注機制,在解碼器中會再經過一次跨專注機制的運算,使用編碼器提供的詢向量和鑰向量對解碼器的值向量進行專注運算。


\begin{figure}
    \centering
    \includegraphics[width=0.9\linewidth]{figures/tfm_arch.drawio.png}
    \caption{轉換器架構圖}
    \label{fig:tfm_arch}
\end{figure}

由於轉換器不需要對每個時間點逐一運算,使其得以實現高度平行化,類神經網路得以透過專注機制同時進行序列資料的大量訓練。這種可擴展性(Scalability)使其在自然語言和語音處理上取得了巨大的進展,近乎取代了原先遞迴式類神經網路的應用場景,近年來甚至被應用在圖像類的資料上\cite{dosovitskiy2021image},展現了此種模型架構的彈性與泛用性,成為目前最前沿的人工智慧主流架構。

除了模型架構,機器學習中不可或缺的另一大部分是對資料的編碼過程。如何更有效率的讓機器理解、處理和輸出資料,是機器學習乃至深層學習的一大課題。面對捉摸不定、抽象且變化萬千的人類語言,語音和文字處理中的表徵學習尤為重要。
\section{表徵與自監督式學習}

\subsection{特徵抽取與表徵學習}
  
不論採用何種模型,為了讓機器可以處理並捕捉輸入資料中的訊號與模式(Pattern),如何對資料編碼和運算的步驟,在機器學習中稱之為特徵抽取(Feature Extraction)或表徵學習(Representation Learning),這是模型建構中不可或缺的重要步驟。

對於抽象的語言概念,早期工程領域根據對語音和文字的理解,分別進行了不同的處理。對於離散且可計數的文字,人們使用詞頻統計衍生出如 n 連詞(n-gram)、TF-IDF(Term-Frequency Inverse Document Frequency)等特徵作為模型學習的前處理步驟;而對於連續且複雜的語音,工程師則透過聲學原理與訊號處理的知識,使用如濾波器組(Filter Bank)、梅爾倒頻譜係數(Mel-Frequency Cepstrum Coefficient,MFCC)等特徵,類比人耳捕捉語音訊號的過程。

在深層學習逐漸發展的過程中,自然語言處理領域的一大里程碑是米氏(Mikolov)提出的「word2vec」模型 \cite{mikolov_efficient_2013},該模型以連續的向量表徵(Vector Representation)取代稀疏(Sparse)的統計數據,對離散的文字單詞進行「詞嵌入(Word Embedding)」編碼。通過大量文本運算,將各單詞之間的共現(Collocation)以跳躍詞(Skip-gram)、連續詞袋(Continuous Bag-of-Word,CBOW)等演算法轉換成高維向量空間中的點,找出每個單詞最適合的語義表徵。爾後,為了更細緻地捕捉同一單詞在不同句子中的脈絡變化,ELMo(Embeddings from Language Model)\cite{peters_deep_2018} 提出了「脈絡化詞嵌入(Contextualized Embedding)」的概念,使得各單詞在運算表徵的過程中可以根據上下文進行些微調整。

\subsection{自監督學習}
  
隨著轉換器模型的提出,BERT(來自轉換器的雙向編碼器表徵,Bidirectional Encoder Representations from Transformers)\cite{devlin_bert_2019} 被提出。通過自專注機制,工程師們無需依賴人工標記,透過預先設定任務(Pretext Task)引導模型從大量文本中自行找出更細緻且考量脈絡(Contextualized)的語義關係,並在許多文字任務上獲得了優異的成績。

自此,楊氏(Yann LeCun)將這種以特定任務作為引導、藉助資料本身的結構替代標註,從大量未標註資料中進行學習資料模式(Pattern)的訓練方式,稱之為「自監督學習(Self-supervised Learning,SSL)」。BERT 的成功使自監督學習得以大行其道,並出現了許多由巨量資料進行預訓練(Pre-train)的基石模型(Foundation Model),有效解決了語言處理領域中的標註資料稀缺的問題。人們在解決語言相關任務時,不需從頭蒐集資料與進行耗時耗能的訓練過程,而是可以利用基石模型優良的泛化(Generalization)能力,解決各種應用任務的需求。相比於預訓練的任務,這些更貼近日常現實的任務被稱為「下游任務(Downstream Task)」,能應對廣泛的下游任務種類,這是基石模型最大的優勢。

有鑑於文字處理方面的成功,語音領域的研究者嘗試將相似模式應用於語音,眾多語音基石模型隨之出現。這些大量的語音資料庫幫助模型萃取出有助於下游任務的語音表徵(Speech Representation),在各種任務上獲得了優於傳統聲學特徵的表現。語音表徵具備的無窮潛力,逐漸成為聲學特徵之外的新選擇。

依照這些語音自監督模型的預訓練學習模式,可大致分為重建式、預測式與對比式模型。以下分別介紹這三類模式:

\subsubsection{重建式學習(Reconstruction Learning)}

此類模型通過對輸入訊號進行擾動(Perturb)後,期望模型將被更動的輸入重新預測回原始資料,通常減損函數表示為:
$$\mathcal{L}_{recon} = \mathbb{E}_x[|f_\theta(\tilde{x}) - x|]$$
其中 $\tilde{x}$ 為擾動後的資料,$f_\theta(\cdot)$ 為模型函數。擾動方式通常以遮蔽為主,在文字處理中以 BERT 為代表,稱為「遮蔽語言模型(Masked Language Model,MLM)」。在語音中,採用此方式學習的有 Mockingjay \cite{liu_mockingjay_2019}、TERA \cite{t_tera_2021} 等模型。  % NPC

\subsubsection{預測式學習(Predictive Learning)}

此類模型通過預訂一些學習目標函數,製造類似輸入與輸出的配對資料,讓模型預測該函數的結果來學習資料中的特定結構。其訓練減損函數可表示為:
$$\mathcal{L}_{pred} = \mathbb{E}_x[\text{eval}(f_\theta(x), \hat{f}(x))]$$
其中 $\hat{f}$ 是期望模型學習的目標函數,$f_\theta(\cdot)$ 為模型函數,$\text{eval}$ 是用來評估預測好壞的標準。

目標函數的典型代表是自迴歸(Autoregressive),期望模型預測未來時間點的輸入表徵。文字方面以生成式預訓練轉換器(Generative Pretrained Transformer,GPT)系列 \cite{radford_language_nodate, brown_language_2020}為代表,語音上的自迴歸預測編碼(Autoregressive Predictive Coding,APC) \cite{chung_generative_2020} 也是採用此種模式。此外,語音基石模型還可以使用其他訓練目標,如 PASE+ \cite{ravanelli_multi-task_2020} 預測其他模型的表徵,而本文著重探究的「隱藏單元 BERT(Hidden-unit BERT,HuBERT)」\cite{hsu_hubert_2021, hsu_hubert_2021-2} 則以預測分群(Cluster)後的輸入表徵為目標,這些預測目標又被視為偽標註(Pseudo-label),後文將著重探討。

\subsubsection{對比式學習(Contrastive Learning)}

此學習方式的訓練目標是要求模型區分正樣本(Positive Sample)與負樣本(Negative Sample)的差異,減損函數通常定義為:
$$\mathcal{L}_{contr} = -\mathbb{E}_x\left[\log
\left(
{\frac
{\sum_{\tilde{x} \in x_{pos}}\exp(\text{sim}(x, \tilde{x}))}
{\sum_{\tilde{x} \in \mathcal{X}}\exp(\text{sim}(x, \tilde{x}))}
}\right)\right]$$

其中 $x$ 為輸入,$x_{pos}$ 為正樣本,$\mathcal{X}$ 為包含正負樣本的資料集,$\text{sim}(\cdot, \cdot)$ 是評估兩個樣本相似程度的函數,常用的相似度函數為內積運算得出的餘弦相似度(Cosine Similarity)。語音上最早使用對比式學習的模型為對比預測編碼(Contrastive Predictive Coding,CPC)\cite{maekaku2022speech},之後如 Wav2vec \cite{schneider2019wav2vec}、Modified CPC \cite{rivière2020unsupervised}、Wav2vec 2.0 \cite{baevski2020wav2vec} 等模型亦是以對比正負樣本的模式訓練,但訓練時正負樣本的定義有所差異,如 Wav2vec 僅以時間維度上相同的向量為正樣本,其餘則將固定時間內的向量皆視為正樣本。

對比式學習通過正負樣本的定義,將預訓練任務形塑為分類問題,因此減損函數本質上為交叉熵,使模型能夠判斷訓練資料中的結構差異。

\subsection{向量量化與離散單元}
  
語音訊號雖然記錄語言資訊,卻與影像資料一樣都是連續數值資料,不像離散的文字較易處理,因此發展出了許多應用廣泛的模型。為了使語音模型訓練可以套用自然語言處理領域的演算法,從連續語音中找出離散表徵逐漸成為研究趨勢,這類研究被稱為「聲學單元發掘(Acoustic Unit Discovery,AUD)」。

由於語言概念本質上是離散符號,向量量化技術常用於涉及語言標註的情境,如電腦視覺經典的量化向量變分自編碼器(Vector-Quantized Variational Autoencoder,VQ-VAE)\cite{van2017neural},利用影像標註的離散語言單詞特性,使模型學習的表徵向量被約束在編碼簿(Codebook) 的幾個向量中。

% 將模型的連續特徵透過甘式軟性最大化和 K-平均兩種向量量化技術
在語音領域,基於 Wav2vec 之上的 Vq-wav2vec \cite{baevski2019vq} 和 Wav2vec 2.0 將連續的語音特徵量化加入訓練目標中,在語音辨識等任務上取得了顯著進步。

HuBERT \cite{hsu_hubert_2021-2} 則應用先對連續的 MFCC 特徵進行 K-平均(K-Means)演算法分群,以所得的群心(Centroid)編號作為訓練目標,實施類似 BERT 的遮蔽語言模型訓練,
並改以此次訓練得到的語音表徵為目標,再次分群後實施第二次訓練。 
這些經過兩輪訓練後,從模型表徵分群得到的群心,被視為「隱藏單元(Hidden Unit)」,編碼了語音訊號中的代表性聲學特徵。透過找出隱藏單元的過程,HuBERT 在低資源情況下達到與 Wav2vec 2.0 相近的語音辨識成績。

\subsection{無文字(Textless)架構}
  
奠基於 HuBERT 等語音基石模型的成功,利用隱藏單元的概念,將大量語音資料表徵進行 K-平均演算法,作為這些語音訊號的偽標籤。如此得到的大量離散隱藏單元形成了「偽文字(Pseudo-text)」的語料庫,基於這些離散單元訓練語言模型,稱為「生成式口語語言模型(Generative Spoken Language Model,GSLM)」\cite{lakhotia_generative_2021-1}。配合反向語音合成訓練基於離散單元的語音生成模型,整體架構不依賴文字標註,訓練出純語音語言模型,稱為「無文字(Textless)架構」\cite{noauthor_textless_2021}。

無文字模式在語音問答(Spoken Question Answering)\cite{lin2022dual}和語音到語音翻譯 (Speech-to-speech Translation)\cite{chen_speech--speech_2023}中取得了前所未有的進展。這些「離散單元(Discrete Unit)」被視為類似文字卻不依賴人類文字標記的語音表徵,具有儲存位元率低和可套用文字語言模型訓練模式的優勢,受到語音社群的廣泛借鑑,後續也帶出了許多如\cite{zhang2024speechtokenizer} 等將語音以離散表徵編碼的研究。

雖然在系統與應用任務上取得了成功,但這些離散單元本身與文字的差異,及其對語音語言模型訓練的幫助,仍是領域內探討的焦點。有鑑於此,本論文基於語言知識,從最接近文字且與語音訊號最相關的「音位(Phoneme)」開始探討,期望了解離散單元能帶來的特徵及其對後續應用的幫助。

\section{本章節總結}
  
本章節首先介紹了深層學習模型的核心部件 --- 類神經網路的基本原理,隨後對本論文研究的核心 --- 「語音表徵」與「離散單元」的發展與歷史進行了梳理。接下來的章節將緊扣這些基石模型得到的離散特徵,對其與「音位」這類語音學標記之間的統計關係進行更深入分析。

% phone type
% alignment
% 最後!處理數據作圖…!! 是要多久……啊就那樣

\chapter{單一語音離散表徵與音位的關係}  % 與語音標記的對應模式

  由於 HuBERT 和 wav2vec 2.0 等語音基石模型的成功,不但在語音任務上達到了前所未有的表現,還促使從語音表徵離散化的想法得以發展。以此產生的「無文字(Textless)」架構,讓人們在處理語音訊號時,有了連續表徵以外的新選擇。離散形式的表徵,可以直接應用文字領域發展的技術,如機器翻譯、生成式模型等,為語音技術帶來新的突破。另一方面,基於離散「符記(token)」的共同形式,離散語音表徵可以更好的整合文字資料,促成多模態領域的發展。跨模態離散表徵的成功,甚至驅使影像領域也開始嘗試發展離散表徵,如探討唇語的 AV-HuBERT \cite{shi2021learning} 等等,展現了離散表徵在資料處理上的優勢。  % 不是模型運算
% 還有影像的 vokenization \cite{tan-bansal-2020-vokenization} 等等。好像更早
% TODO: 找到影像那邊的新的 token?

% 語言學那邊只有一般語音?還是包含 acoustic phon 嗎?
        此外,除了技術的角度切入,為了探討離散語音表徵成功背後的可能因素,以及它們和語言學對人類語音理解之間的差異,甚至是進而得以利用這些技術協助他們更細緻的探討人類的語音現象。因此,原先在連續語音表徵上的語音學分析,也開始關注離散表徵背後有多能描述對語音現象,將其列入考量,成為除了連續語音特徵和時頻譜之外的另一個選擇。
% TODO: cite  "is cont necessary? 那篇"
% 更由於這些離散表徵形式上和音位、文字的相似性,  %% 這個好像還沒?
%% 他們還是更喜歡 IPA,只是先當成一種新的 label 吧?

\section{相關研究}

\subsection{無文字與離散語音表徵}

  自從 HuBERT 帶起來的研究之後,愈來愈多離散表徵相關的研究,例如 \mycite{這邊要 cite 一些東西} 等等。它們在提出自己的離散表徵時,也會採用原先 HuBERT 提供的那些衡量方式,
% 例如 PNMI 等等的,
來驗證這些離散單元確實是與語音中的內容與人類對語音的詮釋具有一定程度的相關性,並從消息理論的角度,證明這些偽標記找出來的符號之間,確實可以做到區分語音中不同形式的資訊。

\subsection{語音學分析}

  另一個層面,由於語音處理本身所針對與探討的終究是人類的語音。因此,有一群研究者通過對人類語音本身的理解,將這些知識應用在分析模型如何對語音訊號建構表徵之上。例如 \mycite{這邊要 cite 那些如唐顥等分析語音本身的} 等。
% 就韋誠說的妹子那篇嗎?

        基於這些作品都對語音的離散表徵感到興趣而做出的探討,本論文也先透過過往幾個常用來分析語音表徵的方式,特別是HuBERT 原始 paper 中提出的標準進行初步的分析。以下介紹此次分析語音表徵的衡量方式:

\section{衡量方式}

  首先是純度(Purity)、熵(Entropy)和相互資訊(Mutual Information,MI),這類標準是在原始 HuBERT 論文中採用的指標 \cite{hsu_hubert_2021, hsu_hubert_2021-2},在比對機器學習過程得到的偽標記與人類知識的標註之間,兩者的相關性(Correlation)。以下對各標準進行詳細解釋:

        不論是什麼語音基石模型,語音表徵的基本單位是音框(Frame)。因此一段語句(Utterance)的語音的離散單元被表示為 $[y_1, \cdots\cdots, y_T]$。其中 $T$ 是該段語句的音框總數。對於該段語句,若給予一段在音框上對齊的語音學標註(Phonetic Label) $[z_1, \cdots\cdots, z_T]$,此時我們可以將離散單元與標註之間配對的出現次數,寫為一個雙變數的共同分佈(Joint Distribution)
    \begin{align}
      p_{yz} = \frac{\sum^T_{t=1}[{y_t = i \wedge z_t = j}]}{T}
    \end{align}

其中 $i$ 是第 $i$ 個音位類別,而 $j$ 指編號為$j$的離散單元。兩個變數的邊際機率(Marginal Probability)分別為
    \begin{align}
        p_z(j)=\sum_i{p_{yz}(i, j)}
    p_y(i)=\sum_j{p_{yz}(i, j)}
\end{align}

因此,對於每一個音位 $i$ 而言,這個音位對應最可能的離散單元為
    \begin{align}
      z^\ast(i) = \arg\max_j p_{yz}(i, j)
    \end{align}
與之相對應的,對於每一個離散單元的類別 $j$ 則可以找到機率最高的音位
    \begin{align}
      y^\ast(j) = \arg\max_i p_{yz}(i,j)
    \end{align}

於是我們可以計算出以下指標:

\subsection{純度}

        本指標考慮音位和離散單元兩個序列之間對應的最高機率,因此從音位與離散單元的角度出發,可以得到以下兩項數據:

\paragraph{音位純度(Phoneme Purity)}

        考慮每個離散單元對應的音位中,最高機率音位的機率,表示為
    \begin{align}
      \mathbb{E}_{p_z(j)}\left[p_{y|z}(y^*(j)|j) \right]
    \end{align}

此指標表示該單元是否對與它對應的音位有足夠的代表性。

\paragraph{分群純度(Cluster Purity)}

        與音位純度相對,改以每個音位的角度,考慮對應單元類別的機率
    \begin{align}
      \mathbb{E}_{p_y(i)}\left[p_{z|y}(z^*(i)|i) \right]
    \end{align}

        由於離散表徵進行分群演算法時的類別數是一項超參數(Hyperparameter),且通常離散單元的分群數量會比音位多,因此該統計數據本身不直接具有語音學的解釋意義,而且在分群數量很多時會顯著下降。

        然而該指標在考量音位純度時必須一併考慮,因為當分群數非常多時,分群純度過低可能使得音位純度相較失去意義。一個極端的情形是每一個音框都給予不同的離散單元編號,如此音位純度可以達到 100\%。但如此一來,離散單元做不到歸納音位類別的目的,音位純度也就失去了意義。

\subsection{熵和相互資訊}

  除了純度提供「最高機率」的對應關係,根據 HuBERT 論文 \cite{hsu_hubert_2021-2} 中的分析方式,我們也可以從資訊理論(Information Theory)的角度,觀察兩個序列的熵和相互資訊。

\paragraph{熵}

  熵的定義按照資訊理論,衡量兩個序列中標籤類別出現機率的不確定性(Uncertainty),公式寫作:

    \begin{align}
        H(y) = \sum_i{p_y(i)\log p_y(i)}
        H(z) = \sum_j{p_z(j)\log p_z(j)}
    \end{align}
式子中 $H(y)$ 和 $H(z)$ 分別為音位和離散單元的熵。

\paragraph{以音位標準化之相互資訊(Phone-normalized Mutual Information,PNMI)}

  本數據以「觀察到某一個離散單元,能降低多少音位標註的不確定性」,定義該離散單元的出現背後提供了多少音位的資訊。公式寫為:
    \begin{align}
        \frac{I(y;z)}{H(y)}&=\cfrac{\sum_i \sum_j p_{yz}(i, j) \log \cfrac{p_{yz}(i, j)}{p_y(i)p_z(j)}}{\sum_i p_y(i) \log p_y(i)} \\
        &=\frac{H(y)-H(y|z)}{H(y)} \\
        &=1-\frac{H(y|z)}{H(y)}
    \end{align}

        該項數據愈高,表示離散單元的分群愈是足以提供語音音位的資訊,是一個品質更好的分群結果。由於離散單元能多好的對應到音位才是人們所關心的問題,因此與純度不同,只以音位的角度出發,而不考慮以離散單元分群的角度。


(Ch 3 好像要講一下為什麼不做連續特徵 \& 為什麼要以 phoneme 為客體了 T\_T) 
\section{動機簡介}  % \section{簡介}



由於 HuBERT 之後,unit 的使用很廣泛,因此為了研究 unit 本身為什麼會被如此適當的可以讓模型視為文字對語音資料進行訓練,我們先從離散表徵本身的特徵分析起。 

由于 HuBERT 模型在使用离散单元进行语音处理方面取得了显著成功,为了研究这些单元为什么能够如此有效地训练模型,我们从离散表征本身的特征分析起。探讨这些单元是否能够替代文字作为语音数据的表征,并分析其与语言学标签之间的关系。 


\section{相關研究}
近期已經有多項相關的研究,嘗試在 SSL 這麼厲害的表現之後找原因,因此有針對 unit 背後 repr 的特性進行分析的 work,例如 CITEUSPLEASE。 

\subsection{語音表徵的語音學分析}
在 HuBERT 出來之後,有一些研究像是 cite 等等,試圖探討對於語音表徵這樣語音模型的基礎進行各種從統計和語音學領域知識角度的分析,以期望能夠解釋為什麼模型可以擁有如此的表現。

此後,cite 等等作品則是從原先連續的表徵出發,開始往離散的量化向量,甚至是離散單元進行分析比對。雖然分析的切入角度可以相當多樣,例如 ABX、tsne 降維分群等等,但本次研究主要著重比對兩者之間在同一段語音序列上給予標籤的相關性,也就是以「偽標籤(pseudo-label)」的角度進行衡量。

\subsection{無文字(textless)語言模型}

這系列 textless 以 GSLM 為最主要代表作,旨在探討 unit 作為一種替代文字的方案。

本論文以 GSLM textless 採用的模型 units 為主要分析對象,企圖銜接兩者的脈絡,來佐證這些 unit 作為一種「類似或可替代文字的語音紀錄方式」在能夠發揮 LM 的特長背後,是否是基於符合語音學特徵帶來的,抑或有什麼其他特徵。



\subsubsection{相关研究}

近期有多项相关研究尝试在 SSL 表现优异之后探讨其原因,例如 Hsu 等人的研究。他们的研究表明,通过使用 k-means 聚类生成的隐藏单元作为预训练的目标标签,HuBERT 模型能够在低资源条件下取得卓越的语音识别性能。这一发现表明,pseudo-label 技术在自监督学习中起到了关键作用。

\paragraph{语音表征的语言学分析}

在 HuBERT 出现之后,一些研究试图从统计和语言学领域知识角度分析语音表征,例如 Wells 等人的研究。这些研究通过分析离散表征与语言学标记的对应关系,探讨模型为何能够如此有效地捕捉语音数据的结构和模式。常见的方法包括 ABX 测试、t-SNE 降维分群等。

\paragraph{无文字(Textless)语言模型}

本论文以 GSLM textless 采用的模型 units 为主要分析对象,企图探讨这些 unit 是否符合语言学特征,或者有其他特征。

 
\section{語音學分類(Phone Type)}

\subsection{簡介}

  除了單一音位本身的特性以外,由於音位本身彼此不是完全獨立的,而是彼此之間就存在相似的特徵,可以分成幾個組別。因此,依照 \mycite{tanghao等三篇} 的分組方式,對英語的音位進行分類並合併比對數據,看看這些離散單元本身是否有擷取到相似的發聲特徵,而不單純只是把音位分成約 50 類完全獨立的標籤。

% (基於語音表徵本身就是 acoustic sisgnals 來的,應該 by nature 要可以對語音特徵分組吧?)

        英語中的音位分為元音(Vowel)與輔音(Consonant)兩大類別,其中又可依照發音的共同特性一共分成七個類別。

\paragraph{元音}

% 母音在這邊為了簡單起見,會被分在一起?

根據發音的位置是否發生改變,英語的元音可分為:

\begin{itemize}
    \item 單元音(Monophthong)
    \item 雙元音(Diphthong)
\end{itemize}

兩大類別。

\paragraph{輔音}

而輔音按照發音的方式,可分為以下五類:

\begin{itemize}
    \item 爆破音 (Plosive)
    \item 擦音(Fricative)
    \item 塞擦音(Affircate)
    \item 鼻音(Nasal)
    \item 近音(Approximant)
\end{itemize}

\subsection{解釋意義}

\begin{itemize}
    \item 純度(Purity):換成以語音學的類別作為新的語音標籤後,有何變化(關聯性更強?)
    \item 熵(Entropy)(放直方圖解釋) \rightarrow 語音分類更明顯?
    \item 對齊(Alignment):是否減少分段資訊的保留(連續子音母音被合併?)
\end{itemize}


\section{實驗集與分析模型}

  本研究的分析對象參考無文字架構 \cite{noauthor_textless_2021, lakhotia_generative_2021, lakhotia_generative_2021-1}  研究,採用當中提及的 CPC、Wav2vec 2.0 和 HuBERT 三個語音基石模型,並與作為比對的聲學特徵梅爾時頻譜(Mel-Spectrogram),一共四種語音表徵模型。

    \begin{align}
        \text{((寫一下 resolution 跟層數))}
    \end{align}

此後亦跟隨該研究選用特定模型層數,和其釋出的K-平均量化模型。這些模型層數與量化為該研究中被證明與語音學特徵最相關,且被使用於無文字架構後續研究之語音離散表徵的抽取方法。無文字研究中已透過 (某某) 語料對四種語音表徵進行 K-平均分群演算法,分別得到群數為 50、100 和 200 的三個量化模型。

        本論文以公開的 LibriSpeech 資料集為分析對象,採取其 train-clean-100 為分析的語音語料庫。
% 因此,
本研究將語音語料庫的語音資料經過四個模型獲取連續表徵後,再經過量化模型得到完全由離散單元組成的「偽文字」語料。

        針對語音學的音位標註,吾人透過 Montreal 強迫對齊器(Forced-Aligner) [[CITE montreal]] 的英語預訓練模型, 從語料庫的文字轉寫 取得 語音資料的音位標註 與對應的時間範圍。 最後透過 語音表徵各自的 時間解析度 生成 以音框為單位的 音位標註語料。最後將兩者對語音資料集進行音位標註相關性的分析。

\section{分析結果}

%% (((還是放一下長條圖說個事兒好了)))     

\subsection{基於各自音位的分析}


        \begin{table}[!htbp]
            \centering
            \begin{subtable}[t]{\textwidth}
                \centering
                \begin{tabular}{cccccc}
                                & 音位純度 & 分群純度 & 音位熵 & 離散單元熵 &    PNMI \\
                    HuBERT      &   0.5256 &   0.3382 & 3.3152 &     3.8681 & 0.4993 \\   %% 1.6552 h
                    wav2vec 2.0 &   0.4006 &   0.2676 & 3.3152 &     3.8215 & 0.3706 \\   %% 1.2286 w
                    CPC         &   0.5188 &   0.3812 & 3.3146 &     3.7918 & 0.4992 \\   %% 1.6545 c
                    LogMel      &   0.3253 &   0.1473 & 3.3158 &     3.8630 & 0.2647 \\   %% 0.8776 l
                \end{tabular}
                \caption{群數 = 50}
                \label{tab:ch3-clu050}
            \end{subtable}        

            \vspace{0.5cm}        

            \begin{subtable}[t]{\textwidth}
                \centering
                \begin{tabular}{cccccc}
                                & 音位純度 & 分群純度 & 音位熵 & 離散單元熵 &    PNMI \\
                    HuBERT      &   0.6097 &   0.2553 & 3.3152 &     4.5704 & 0.5786 \\   %% 1.9181 h
                    wav2vec 2.0 &   0.4877 &   0.2118 & 3.3152 &     4.5284 & 0.4596 \\   %% 1.5235 w
                    CPC         &   0.5895 &   0.2674 & 3.3146 &     4.5034 & 0.5557 \\   %% 1.8418 c
                    LogMel      &   0.3348 &   0.0931 & 3.3158 &     4.5591 & 0.2789 \\   %% 0.9247 l
                \end{tabular}
                \caption{群數 = 100}
                \label{tab:ch3-clu100}
            \end{subtable}        

            \vspace{0.5cm}        

            \begin{subtable}[t]{\textwidth}
                \centering
                \begin{tabular}{cccccc}
                                & 音位純度 & 分群純度 & 音位熵 & 離散單元熵 &    PNMI \\
                    HuBERT      &   0.6474 &   0.1644 & 3.3152 &     5.2681 & 0.6289 \\   %% 2.0849 h
                    wav2vec 2.0 &   0.5427 &   0.1467 & 3.3152 &     5.2173 & 0.5188 \\   %% 1.7199 w
                    CPC         &   0.6098 &   0.1789 & 3.3146 &     5.1885 & 0.5882 \\   %% 1.9497 c
                    LogMel      &   0.3474 &   0.0569 & 3.3158 &     5.2322 & 0.2955 \\   %% 0.9798 l
                \end{tabular}
                \caption{群數 = 200}
                \label{tab:ch3-clu200}
            \end{subtable}        

            \caption{不同群數在四種基石模型的分析數據}
            \label{tab:single-cluster-results}
        \end{table}

  由表 \ref{tab:single-cluster-results} 中可以看出,分群的群數愈多時,音位的純度確實有所上升,但這可能是犧牲分群純度得來的。因此再看 PNMI 的指標可以發現,整體離散單元和音位標註的相關性還是有所提升的。

        此外,就不同模型來觀察,HuBERT 的表現是四種語音表徵之中最好的,一定程度上可以佐證 HuBERT 在找出語音中有意義單位上的效能,以及為什麼無文字架構通常以 HuBERT 當成抽取語音離散表徵的模型。

\subsection{基於語音學分類的分析}

        \begin{table}[!htbp]
            \centering
            \begin{subtable}[t]{\textwidth}
                \centering
                \begin{tabular}{cccccc}
                                & 音位純度 & 分群純度 & 音位熵 & 離散單元熵 &    PNMI \\
                    HuBERT      &   0.5256 &   0.3382 & 3.3152 &     3.8681 & 0.4993 \\   %% 1.6552 h
                    wav2vec 2.0 &   0.4006 &   0.2676 & 3.3152 &     3.8215 & 0.3706 \\   %% 1.2286 w
                    CPC         &   0.5188 &   0.3812 & 3.3146 &     3.7918 & 0.4992 \\   %% 1.6545 c
                    LogMel      &   0.3253 &   0.1473 & 3.3158 &     3.8630 & 0.2647 \\   %% 0.8776 l
                \end{tabular}
                \caption{群數 = 50}
                \label{tab:ch3-clu050}
            \end{subtable}        

            \vspace{0.5cm}        

            \begin{subtable}[t]{\textwidth}
                \centering
                \begin{tabular}{cccccc}
                                & 音位純度 & 分群純度 & 音位熵 & 離散單元熵 &    PNMI \\
                    HuBERT      &   0.6097 &   0.2553 & 3.3152 &     4.5704 & 0.5786 \\   %% 1.9181 h
                    wav2vec 2.0 &   0.4877 &   0.2118 & 3.3152 &     4.5284 & 0.4596 \\   %% 1.5235 w
                    CPC         &   0.5895 &   0.2674 & 3.3146 &     4.5034 & 0.5557 \\   %% 1.8418 c
                    LogMel      &   0.3348 &   0.0931 & 3.3158 &     4.5591 & 0.2789 \\   %% 0.9247 l
                \end{tabular}
                \caption{群數 = 100}
                \label{tab:ch3-clu100}
            \end{subtable}        

            \vspace{0.5cm}        

            \begin{subtable}[t]{\textwidth}
                \centering
                \begin{tabular}{cccccc}
                                & 音位純度 & 分群純度 & 音位熵 & 離散單元熵 &    PNMI \\
                    HuBERT      &   0.6474 &   0.1644 & 3.3152 &     5.2681 & 0.6289 \\   %% 2.0849 h
                    wav2vec 2.0 &   0.5427 &   0.1467 & 3.3152 &     5.2173 & 0.5188 \\   %% 1.7199 w
                    CPC         &   0.6098 &   0.1789 & 3.3146 &     5.1885 & 0.5882 \\   %% 1.9497 c
                    LogMel      &   0.3474 &   0.0569 & 3.3158 &     5.2322 & 0.2955 \\   %% 0.9798 l
                \end{tabular}
                \caption{群數 = 200}
                \label{tab:ch3-clu200}
            \end{subtable}        

            \caption{不同群數在四種基石模型的分析數據}
            \label{tab:single-cluster=phonetype-results}
        \end{table}


\section{本章總結}

  本章節探討以音框為單位取出的語音離散表徵與對應的音位標註之間的關係,從分析結果中可以得到,HuBERT 模型的離散表徵確實與人類理解的語音單位「音位」之間,具有最明顯的相似性,也進一步證明
% 以此尋找語音中的
為何 HuBERT 目前是抽取語音離散表徵時最常使用的模型。


% 
\section{動機}

基於單一語音單元表徵的限制,我們

放一些 rate 的東西



\section{相關研究}

\subsection{對語音離散表徵的分詞(tokenization)研究}


% 
\section{分詞方法}

在文字語言模型上常用的分詞方法是 BPE,這種分詞方法比較於原本的 word 有幾個好處:

1 可以



\section{衡量方式}

\subsection{字符(token)與音位之間的關係}

\subsection{壓縮比率}

% 


\section{分析結果}


\subsection{基於各自音位的分析}

在此部分,我们将展示基于音位的频率统计结果和纯度分析结果。通过这些分析,我们可以了解离散单元在捕捉音位特征方面的表现 (\href{https://www.isca-archive.org/interspeech_2022/ren22_interspeech.html}{ISCA Archive}) (\href{https://deepai.org/publication/speech-pre-training-with-acoustic-piece}{DeepAI})。

\subsection{基於語音學分類的分析}


在此部分,我们将展示基于语音学分类的分析结果。这包括对离散单元在不同语音学类别(如元音、辅音)中的表现分析,以评估其在不同语音特征下的表现 (\href{https://www.isca-archive.org/interspeech_2022/ren22_interspeech.html}{ISCA Archive}) (\href{https://deepai.org/publication/speech-pre-training-with-acoustic-piece}{DeepAI})。



 
% 
\section{應用在語音任務的實驗}

\subsection{語音辨識}

\subsubsection{實驗設定與資料集}

\subsubsection{實驗結果與其和分析數據間的關係}



\section{本章總結}


% \chapter{結論與展望}\section{研究貢獻與討論}\section{未來展望}

% \input{thesis_backpages}

% <<<
\newpage
\phantomsection % for hyperref to register this
\addcontentsline{toc}{chapter}{\nameRef}
\renewcommand{\bibname}{\protect\makebox[5cm][s]{\nameRef}}
%  \makebox{} is fragile; need protect
\bibliographystyle{IEEEtran}  % 使用 IEEE Trans 期刊格式
\bibliography{thesis}

% >>>

\end{document}
