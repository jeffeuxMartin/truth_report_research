% 20B-21A
% 21B-22A
% 22B-23A
% 23B-24A

\section{相關研究 (NOT YET)}  
\subsection{(Version MINE)}
  
這類研究大約在無文字架構被提出後的約兩年內逐步出現,但各篇研究對於「應用次詞(Subword)概念於離散單元」的切入點都有所差異。

首先\cite{ren_speech_2022}\citetag{22A4-Pretrain-ap}基於離散單元與音位的關聯性,並透過觀察到離散單元序列中,對應轉寫文字的單詞,兩者互相比對可以觀察到不少相似的模式(Pattern),甚至在不同語者之間也能觀察到此一現象。以此本篇論文首先將離散單元使用 SentencePiece 分詞演算法後的新單元「聲學片段(Acoustic Piece)」用於語音辨識的訓練上。

近乎同時,\cite{wu_wav2seq_2023}\citetag{22A5-wav2seq}基於長度考量,以及離散單元和音位的關聯性,首先將離散單元視為字符(Character),嘗試將這些字符透過分詞方法組成「偽語言(Pseudo-language\footnote{對應於離散單元被視為「偽文字(Pseudo-text)」})」,來幫助語音到文字的模型。因為解碼器在實際應用時需要生成的序列多是文字的符記 --- 次詞單位,因此該篇研究旨在讓模型在預訓練時可以更適應下游任務。

\uucite{23A-coarser-grain} 

近期,\cite{chang_exploring_2024}\citetag{23B-shinji-hsiuhsuan}將次詞單位分詞的作法納入 ESPNet 套件在處理離散單元相關語音任務的固定流程(Pipeline)之中,並獲得了超越以往的表現,進一步證明了這個方法的效果。
% speech generation
















\myhline




% \textbf{\#\# 對語音離散表徵的分詞研究}

\subsection{(Version ChatGPT)}

% \section{分詞方法在語音處理中的應用}

在語音處理領域,這些分詞方法同樣適用。通過對離散語音單元的分詞處理,可以有效地將語音信號轉換為子詞單元,提高語音識別和翻譯模型的性能。例如,Ren 等人的方法通過句子片段化處理將高頻代碼模式合併為聲學片段,從而有效地將輸入音頻與自然語言橋接起來 \cite{ren_speech_2022}。
