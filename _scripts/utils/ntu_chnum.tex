% this file is encoded in utf-8
% v3.0 (Jun. 11, 2019)

% 請依需要選擇其中一種表現方式,把它所對應的指令列打開,其他沒有用到的表現方式的對應指令列請關閉。(用行首百分號)

%% 第一種目錄格式:
%%	1  簡介 ............................ 1
%%
%%      章別 (chapter counter) 「1」前後沒有其他文字,
%%
%%      內文章標題是
%%		第 1 章	簡介
%%	\tocprechaptername, \tocpostchaptername 都設成沒有內容的空字串
%%	\tocChNumberWidth 設成 1.4em (預設)
%%      底下三行指令請打開
%\renewcommand\tocprechaptername{}
%\renewcommand\tocpostchaptername{}
%\setlength{\tocChNumberWidth}{1.4em}


%% 第二種目錄格式:
%%	一、簡介 ............................ 1
%%
%%      章別 (chapter counter) 「一」前沒有文字,後有頓號,
%%
%%      內文章標題是
%%		第一章		簡介
%%	\tocprechaptername 設成沒有內容的空字串
%%	\tocpostchaptername 設成頓號
%%	\tocChNumberWidth 設成 2em
%%      底下四行指令請打開 (預設)
\renewcommand\countermapping[1]{\zhnumber{#1}}
\renewcommand\tocprechaptername{}
\renewcommand\tocpostchaptername{、}
\setlength{\tocChNumberWidth}{2em}


%% 第三種目錄格式:
%%	第一章、簡介 ......................... 1
%%
%%      章別 (chapter counter) 「一」前有「第」,後有「章」與頓號,
%%      內文章標題是
%%		第一章		簡介
%%	\tocprechaptername 設成「第」
%%	\tocpostchaptername 設成「章、」
%%	\tocChNumberWidth 設成 3em
%%      底下四行指令請打開
%\renewcommand\countermapping[1]{\CJKnumber{#1}}
%\renewcommand\tocprechaptername{第}
%\renewcommand\tocpostchaptername{章、}
%\setlength{\tocChNumberWidth}{3em}



%% 可以依照需要作彈性的設定
%%
%% 章別 (數字,包括後面的字串) 的寬度 \tocChNumberWidth,
%% 會影響章名與章別之間的間隔 (太少則相疊,太多則留白)
%% 建議設成 \tocpostchaptername 內容字數加一,做為 em 的倍數,
%% 但至少也要有 1.4 倍。
