%%%% Common Symbol %%%%
\newcommand*{\Nb}{\mathbb{N}}
\newcommand*{\Zb}{\mathbb{Z}}
\newcommand*{\Qb}{\mathbb{Q}}
\newcommand*{\Rb}{\mathbb{R}}
\newcommand*{\Cb}{\mathbb{C}}
\newcommand*{\Eb}{\mathbb{E}}

\newcommand*{\Ac}{\mathcal{A}}
\newcommand*{\Dc}{\mathcal{D}}
\newcommand*{\Oc}{\mathcal{O}}
\newcommand*{\Tc}{\mathcal{T}}
\newcommand*{\Uc}{\mathcal{U}}
\newcommand*{\Vc}{\mathcal{V}}
\newcommand*{\Xc}{\mathcal{X}}
\newcommand*{\Yc}{\mathcal{Y}}
\newcommand*{\Zc}{\mathcal{Z}}
\newcommand*{\trans}{\mathsf{T}}

%%%% mapping Symbol %%%%
\newcommand*\bij{\lhook\joinrel\twoheadrightarrow}
\newcommand*\oneto{\hookrightarrow}
\newcommand*\onto{\twoheadrightarrow}
\newcommand*\isoto{\xrightarrow{\sim}}
\newcommand*\acts{\curvearrowright}
\newcommand*\revacts{\curvearrowleft}

%%%% set definition %%%%
% just to make sure it exists
\providecommand\given{}
% can be useful to refer to this outside \Set
\newcommand*\SetSymbol[1][]{%
  \nonscript\:#1\vert
  \allowbreak
  \nonscript\:
\mathopen{}}
\DeclarePairedDelimiterX\Set[1]\{\}{%
  \renewcommand\given{\SetSymbol[\delimsize]}
  \,#1\,
}

%%%% Probability and Statistics Notations %%%%
\DeclarePairedDelimiterXPP{\KL}[2]{D_\text{KL}}\lbrack\rbrack{}{{#1} \delimsize\Vert {#2}}
\DeclarePairedDelimiterXPP{\Prob}[1]{\Pb}\lbrace\rbrace{}{#1}
\DeclarePairedDelimiterXPP{\Ev}[1]{\Eb}\lbrack\rbrack{}{#1}
\DeclarePairedDelimiterXPP{\Evr}[2]{\Eb_{#1}}\lbrack\rbrack{}{#2}

%%%%%%%%symbol and function settings%%%%%%%%%
\DeclarePairedDelimiter{\abs}{\lvert}{\rvert}
\DeclarePairedDelimiter{\norm}{\lVert}{\rVert}
\DeclarePairedDelimiter{\inpd}{\langle}{\rangle} % inner product
\DeclarePairedDelimiter{\ceil}{\lceil}{\rceil}
\DeclarePairedDelimiter{\floor}{\lfloor}{\rfloor}
\DeclareMathOperator*{\argmin}{arg\,min}
\DeclareMathOperator*{\argmax}{arg\,max}
\newcommand*{\bv}[1]{\mathbf{#1}}

%% some function name
\DeclareMathOperator{\ELBO}{ELBO}
\DeclareMathOperator{\ReLU}{ReLU}

