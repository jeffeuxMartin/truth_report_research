
\section{動機}

基於單一語音單元表徵的限制,我們

放一些 rate 的東西


在现有的语音表征方法中,单一的离散单元虽然在许多任务中表现良好,但在处理长序列和时间分辨率方面存在局限性。Ren 等人的研究表明,许多相似的模式是以连续几个单元的形式出现的,这提示我们可以通过组合多个离散单元来更好地表示语音信号,从而提高模型在语音识别和语音翻译等任务中的性能 (\href{https://www.isca-archive.org/interspeech_2022/ren22_interspeech.html}{ISCA Archive}) (\href{https://deepai.org/publication/speech-pre-training-with-acoustic-piece}{DeepAI})。此外,SpeechTokenizer 的研究进一步展示了通过分层建模不同信息层次,能够在语音表示中捕捉到更丰富的语义和语音学信息 (\href{https://ar5iv.org/pdf/2308.16692}{ar5iv})。 


\section{相關研究}

\subsection{對語音離散表徵的分詞(tokenization)研究}

Ren 等人提出了使用声学片段(acoustic pieces)作为预训练的目标标签,通过对 HuBERT 代码进行句子片段化处理得到,能够显著改善自动语音识别(ASR)的性能 (\href{https://www.isca-archive.org/interspeech_2022/ren22_interspeech.html}{ISCA Archive})。Wu 等人介绍了 Wav2Seq,这是一种自监督方法,通过伪语言(pseudo languages)预训练编码器和解码器,显著提升了 ASR 和语音到文本翻译的性能 (\href{https://deepai.org/publication/speech-pre-training-with-acoustic-piece}{DeepAI})。

BEATs 提出了音频预训练方法,通过声学分片(acoustic tokenizers)学习音频离散表示,用于各种音频分类任务,表现出色 (\href{https://ar5iv.org/abs/2212.09058}{ar5iv})。

SpeechTokenizer 利用层级建模和语义蒸馏技术,通过多层残差量化器(RVQ)捕捉不同层次的信息,从而在语音表示中实现更高的语义和语音学特征保真度 (\href{https://ar5iv.org/pdf/2308.16692}{ar5iv})。

