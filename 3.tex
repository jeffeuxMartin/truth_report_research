\chapter{單一語音離散表徵}  % 與語音標記的對應模式
  
由於 HuBERT 與 textless 架構的廣泛發展與突破,帶起了語音中離散表徵的研究。
相比於連續表徵,離散表徵的「token」形式,
一來讓它可以更好的應用文字處理領域的各種如翻譯、生成式模型等技術,
二來可以讓它和文字的之間做到更好的 integration,促成多模態領域的發展,
甚至影像相關的領域也開始出現離散表徵的研究,
% 確認 AV-HuBERT 是不是唇語
如處理唇語等的 AV-HuBERT \cite{shi2021learning} 等,
% 還有影像的 vokenization \cite{tan-bansal-2020-vokenization} 等等。
% 影像好像更早
% TODO: 找到影像那邊的新的 token?
展現了離散表徵在模型處理的優勢。

此外,除了從應用、技術面切入的語音處理工程角度,
為了解釋、探討究竟這些工程技術如何得以成功,
以及它們和語言學對人類語音理解之間的差異,
% 只有一般語音?還是包含 acoustic phon 嗎?
甚至是進而得以利用這些技術協助他們更細緻的探討人類的語音現象。
因此,
原先在連續語音表徵上的語音學分析,
也開始關注離散表徵背後% is cont necessary? 那篇
有多能描述對語音現象,
將其列入考量,
成為除了連續語音特徵和時頻譜之外的另一個選擇。
% 更由於這些離散表徵形式上和音位、文字的相似性,  %% 這個好像還沒?
%% 他們還是更喜歡 IPA
%% 只是先當成一種新的 label 吧?

